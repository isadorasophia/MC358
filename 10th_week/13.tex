%=====================================================================
% jhwhw.cls
% Provide jhwhw.cls class
%=====================================================================

%=====================================================================
% Identification
%=====================================================================

\documentclass[12pt, letterpaper]{report}

\usepackage[utf8]{inputenc}

\usepackage{graphicx}
\usepackage{fancyhdr}
\usepackage[top=1in,bottom=1in,left=1in,right=1in]{geometry}
\usepackage{empheq}
\usepackage{ifthen}

\usepackage{enumitem}

\usepackage{mathrsfs}

\usepackage{amssymb}
\newcommand*{\CQD}{\hfill\ensuremath{\blacksquare}}%


%=====================================================================
% Commands
%=====================================================================

  \setlength{\headheight}{15pt}
  \lhead{Isadora S. G. Rodopoulos}\chead{MC358}\rhead{10 de Maio, 2016}
  \lfoot{}\cfoot{\thepage}\rfoot{}
  \pagestyle{fancy}

\ifx\pdfoutput\undefined                         %LaTeX
  \RequirePackage[ps2pdf,bookmarks=true]{hyperref}
  \hypersetup{ %
    pdfauthor   = {\@author},
    pdftitle    = {\@title},
    pdfcreator  = {LaTeX with hyperref package},
    pdfproducer = {dvips + ps2pdf}
  }
\else                                            %PDFLaTeX
  \RequirePackage[pdftex,bookmarks=true]{hyperref}
  \hypersetup{ %
    pdfauthor   = {\@author},
    pdftitle    = {\@title},
    pdfcreator  = {LaTeX with hyperref package},
    pdfproducer = {dvips + ps2pdf}
  }
\pdfadjustspacing=1
\fi

% Set up counters for problems and subsections

\newcounter{ProblemNum}
\newcounter{SubProblemNum}[ProblemNum]

\renewcommand{\theProblemNum}{\arabic{ProblemNum}}
\renewcommand{\theSubProblemNum}{\alph{SubProblemNum}}


\newcommand*{\anyproblem}[1]{\newpage\subsection*{#1}}
\newcommand*{\problem}[1]{\stepcounter{ProblemNum} %
   \anyproblem{Questão #1}}
\newcommand*{\soln}[1]{\subsubsection*{#1}}
\newcommand*{\solution}{\soln{Solução}}
\renewcommand*{\part}{\stepcounter{SubProblemNum} %
  \soln{Parte (\theSubProblemNum)}}

\renewcommand{\theenumi}{(\alph{enumi})}
\renewcommand{\labelenumi}{\theenumi}
\renewcommand{\theenumii}{\roman{enumii}}

% \def\problemmark{}

% % Typesetting problems

% % \newcommand*{\prob}[1]{\newpage \noindent \textbf{\Large #1}}
% % \newcommand*{\problem}[1]{\stepcounter{ProblemNum} \prob{Problem %
% % \theProblemNum.}}
% % \newcommand*{\soln}[1]{\\ \noindent \textbf{\Large #1}}
% % \newcommand*{\solution}{\soln{Solution}}
% % \renewcommand*{\part}{\\ \noindent \stepcounter{SubProblemNum} %
% % \textbf{\Large Part (\theSubProblemNum)}}

% \newcommand\problem{\@startsection{problem}{1}{\z@}%
%                      {-3.25ex \@plus -1ex \@minus -.2ex}%
%                      {1.5ex \@plus .2ex}%
%                      {\normalfont\large\bfseries}}

\begin{document}

\problem{13.}
  O jogo da Torre de Hanoi consiste em mover um conjunto de discos de um pino para outro (de um total de três pinos), onde é permitido mover apenas um disco por vez, e nenhum disco pode ser posto sobre um de diâmetro menor que o seu próprio.

  Prove que o número de passos necessários para mover $ n $ discos de um pino para outro (qualquer um dos dois restantes) é $ 2^n - 1 $.

\solution
  Prova por indução em $n$:

  \textbf{Passo:} Se temos $n = 1$, então basta mover este disco para o outro pino e foram necessários $2^1 - 1 = 1$ passos. \CQD

  \textbf{Hipótese:} O número de passos necessários para mover $ n $ discos de um pino para outro (qualquer um dos dois restantes) é $ 2^n - 1 $.

  \textbf{Passo:} Suponha que temos que temos $n + 1$ discos. Basta mover $n$ discos menores para o outro pino - o qual, P.H.I., será necessário $2^n - 1$ passos (o pino com o disco de maior diâmetro estará livre para jogar e, portanto, se aplica à hipótese de indução). 

  Assim, o jogo estará disposto com um pino com um disco de diâmetro grande, outro pino com $n$ discos menores e um pino livre restante. Se movermos o maior disco para o pino livre, será necessário mais $2^n - 1$ passos para mover os $n$ discos menores para o pino restante com o disco maior.

  Logo, foram necessários $2^n - 1 + 1 + 2^n - 1 = 2^n + 2^n - 1 = 2^{n + 1} - 1$ passos para mover os $n + 1$ discos de um pino para outro. \CQD

\end{document}
