%=====================================================================
% jhwhw.cls
% Provide jhwhw.cls class
%=====================================================================

%=====================================================================
% Identification
%=====================================================================

\documentclass[12pt, letterpaper]{report}

\usepackage[utf8]{inputenc}

\usepackage{graphicx}
\usepackage{fancyhdr}
\usepackage[top=1in,bottom=1in,left=1in,right=1in]{geometry}
\usepackage{empheq}
\usepackage{ifthen}

\usepackage{enumitem}

\usepackage{mathrsfs}

\usepackage{amssymb}
\newcommand*{\CQD}{\hfill\ensuremath{\blacksquare}}%

%=====================================================================
% Commands
%=====================================================================

  \setlength{\headheight}{15pt}
  \lhead{Isadora S. G. Rodopoulos}\chead{MC358}\rhead{31 de Maio, 2016}
  \lfoot{}\cfoot{\thepage}\rfoot{}
  \pagestyle{fancy}

\ifx\pdfoutput\undefined                         %LaTeX
  \RequirePackage[ps2pdf,bookmarks=true]{hyperref}
  \hypersetup{ %
    pdfauthor   = {\@author},
    pdftitle    = {\@title},
    pdfcreator  = {LaTeX with hyperref package},
    pdfproducer = {dvips + ps2pdf}
  }
\else                                            %PDFLaTeX
  \RequirePackage[pdftex,bookmarks=true]{hyperref}
  \hypersetup{ %
    pdfauthor   = {\@author},
    pdftitle    = {\@title},
    pdfcreator  = {LaTeX with hyperref package},
    pdfproducer = {dvips + ps2pdf}
  }
\pdfadjustspacing=1
\fi

% Set up counters for problems and subsections

\newcounter{ProblemNum}
\newcounter{SubProblemNum}[ProblemNum]

\renewcommand{\theProblemNum}{\arabic{ProblemNum}}
\renewcommand{\theSubProblemNum}{\alph{SubProblemNum}}


\newcommand*{\anyproblem}[1]{\newpage\subsection*{#1}}
\newcommand*{\problem}[1]{\stepcounter{ProblemNum} %
   \anyproblem{Questão #1}}
\newcommand*{\soln}[1]{\subsubsection*{#1}}
\newcommand*{\solution}{\soln{Solução}}
\renewcommand*{\part}{\stepcounter{SubProblemNum} %
  \soln{Parte (\theSubProblemNum)}}

\renewcommand{\theenumi}{(\alph{enumi})}
\renewcommand{\labelenumi}{\theenumi}
\renewcommand{\theenumii}{\roman{enumii}}

% \def\problemmark{}

% % Typesetting problems

% % \newcommand*{\prob}[1]{\newpage \noindent \textbf{\Large #1}}
% % \newcommand*{\problem}[1]{\stepcounter{ProblemNum} \prob{Problem %
% % \theProblemNum.}}
% % \newcommand*{\soln}[1]{\\ \noindent \textbf{\Large #1}}
% % \newcommand*{\solution}{\soln{Solution}}
% % \renewcommand*{\part}{\\ \noindent \stepcounter{SubProblemNum} %
% % \textbf{\Large Part (\theSubProblemNum)}}

% \newcommand\problem{\@startsection{problem}{1}{\z@}%
%                      {-3.25ex \@plus -1ex \@minus -.2ex}%
%                      {1.5ex \@plus .2ex}%
%                      {\normalfont\large\bfseries}}

\begin{document}

\problem{3.}
Prove por indução que se um grafo simples tem pelo menos dois vértices, então ele possui dois vértices de mesmo grau. Explicite em que parâmetro você fez sua indução e se sua prova é por indução fraca ou forte.

\solution
  Prova por indução fraca em $v$, com $v =$ número de vértices.

  \textbf{Base:} Se um grafo simples possui $v = 2$, então ou $e = 1$, e ambos os vértices possuem o mesmo grau $= 1$, ou $e = 0$, e o grau de ambos os vértices $= 0$. \CQD

  \textbf{Hipótese:} Se um grafo simples possui $v \geq 2$ vértices, então ele possui dois vértices de mesmo grau.

  \textbf{Passo:} Suponha um grafo simples $G$ com $n$ vértices, ou seja, ele possui (no máximo) $n - 1$ diferentes graus, uma vez que P.H.I., pelo menos um par possui o mesmo grau.

  Adicionamos mais um vértice $a$. 

  Se o vértice tiver $g(a) = 0$, então o par permanece intacto e o grafo de $n + 1$ vértices possui pelo menos um par com o mesmo grau.
 
  Caso $g(a) = 1$, então: 

  \begin{enumerate}
    \item $a$ está conectado com um vértice $x$ de $g(x) = 1$, e o grafo de $n + 1$ vértices possui pelo menos um par com o mesmo grau.
    \item $a$ está conectado com um dos vértices $x$ que, anteriormente, tinha o mesmo grau de outro vértice.
      \begin{enumerate}
        \item $G$ possuía mais de um par de mesmo grau, e o grafo de $n + 1$ vértices possui pelo menos um par com o mesmo grau.
        \item $G$ possuía apenas um par de mesmo grau. Suponha que $g(x) = d \leq n - 1$ antes da inclusão do novo vértice.
          \begin{enumerate}
            \item Se haver outro vértice de grau $d + 1$, então temos um novo par para o grafo de $n + 1$ vértices.
            \item Se não havia outro vértice de grau $d + 1$, então havia um vértice de grau 1. Ou seja, temos um novo par de grau $1$ para o grafo de $n + 1$ vértices.
          \end{enumerate}
      \end{enumerate}
  \end{enumerate}

  Analogamente, os passos podem ser repetidos para todo número de vértices que $a$ adquirir. Logo, o grafo de $n + 1$ vértices sempre possuirá pelo menos um par de vértices de mesmo grau. \CQD

\end{document}
