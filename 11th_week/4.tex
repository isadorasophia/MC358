%=====================================================================
% jhwhw.cls
% Provide jhwhw.cls class
%=====================================================================

%=====================================================================
% Identification
%=====================================================================

\documentclass[12pt, letterpaper]{report}

\usepackage[utf8]{inputenc}

\usepackage{graphicx}
\usepackage{fancyhdr}
\usepackage[top=1in,bottom=1in,left=1in,right=1in]{geometry}
\usepackage{empheq}
\usepackage{ifthen}

\usepackage{enumitem}

\usepackage{amssymb}
\newcommand*{\CQD}{\hfill\ensuremath{\blacksquare}}%

%=====================================================================
% Commands
%=====================================================================

  \setlength{\headheight}{15pt}
  \lhead{Isadora S. G. Rodopoulos}\chead{MC358}\rhead{31 de Maio, 2016}
  \lfoot{}\cfoot{\thepage}\rfoot{}
  \pagestyle{fancy}

\ifx\pdfoutput\undefined                         %LaTeX
  \RequirePackage[ps2pdf,bookmarks=true]{hyperref}
  \hypersetup{ %
    pdfauthor   = {\@author},
    pdftitle    = {\@title},
    pdfcreator  = {LaTeX with hyperref package},
    pdfproducer = {dvips + ps2pdf}
  }
\else                                            %PDFLaTeX
  \RequirePackage[pdftex,bookmarks=true]{hyperref}
  \hypersetup{ %
    pdfauthor   = {\@author},
    pdftitle    = {\@title},
    pdfcreator  = {LaTeX with hyperref package},
    pdfproducer = {dvips + ps2pdf}
  }
\pdfadjustspacing=1
\fi

% Set up counters for problems and subsections

\newcounter{ProblemNum}
\newcounter{SubProblemNum}[ProblemNum]

\renewcommand{\theProblemNum}{\arabic{ProblemNum}}
\renewcommand{\theSubProblemNum}{\alph{SubProblemNum}}


\newcommand*{\anyproblem}[1]{\newpage\subsection*{#1}}
\newcommand*{\problem}[1]{\stepcounter{ProblemNum} %
   \anyproblem{Questão #1}}
\newcommand*{\soln}[1]{\subsubsection*{#1}}
\newcommand*{\solution}{\soln{Solução}}
\renewcommand*{\part}{\stepcounter{SubProblemNum} %
  \soln{Parte (\theSubProblemNum)}}

\renewcommand{\theenumi}{(\alph{enumi})}
\renewcommand{\labelenumi}{\theenumi}
\renewcommand{\theenumii}{\roman{enumii}}

% \def\problemmark{}

% % Typesetting problems

% % \newcommand*{\prob}[1]{\newpage \noindent \textbf{\Large #1}}
% % \newcommand*{\problem}[1]{\stepcounter{ProblemNum} \prob{Problem %
% % \theProblemNum.}}
% % \newcommand*{\soln}[1]{\\ \noindent \textbf{\Large #1}}
% % \newcommand*{\solution}{\soln{Solution}}
% % \renewcommand*{\part}{\\ \noindent \stepcounter{SubProblemNum} %
% % \textbf{\Large Part (\theSubProblemNum)}}

% \newcommand\problem{\@startsection{problem}{1}{\z@}%
%                      {-3.25ex \@plus -1ex \@minus -.2ex}%
%                      {1.5ex \@plus .2ex}%
%                      {\normalfont\large\bfseries}}

\begin{document}

\problem{4.}
    Prove por indução que se $G$ é um grafo simples bipartido com $v$ vértices e $e$ arestas, então $e \le \cfrac{v^2}{4}$. Explicite em que parâmetro você fez sua indução e se sua prova é por indução fraca ou forte.

\solution
    Prova por indução forte em $v$.

    \textbf{Base:} Para um grafo bipartido com $v = 0$, ele não possui nenhuma aresta e $0 \leq 0$. \CQD

    \textbf{Hipótese:} Seja $G$ um grafo simples bipartido com $v > k \geq 0$ vértices e $e$ arestas, então $e \le \cfrac{v^2}{4}$.

    \textbf{Passo:} Suponha um grafo bipartido de $v$ vértices. Divida-o em dois grafos bipartidos de $k$ e $v - k$ vértices, extinguindo-se qualquer aresta entre os dois grafos. P.H.I., o grafo de $k$ vértices terá, no máximo, $\cfrac{k^2}{4}$ arestas. O grafo de $v - k$ vértices terá, no máximo $\cfrac{(v - k)^2}{4}$.
    Ao juntá-los novamente, temos como máximo de arestas: a quantidade de arestas dos dois grafos $+ k(v - k)$, que corresponde às arestas possivelmente eliminadas na divisão do grafo original.
    
    Assim, temos $\cfrac{k^2}{4} + \cfrac{(v - k)^2}{4} + k(v - k) = \cfrac{k^2 + v^2 - 2vk + k^2 + 4kv - 4k^2}{4} = \cfrac{-2k^2 + v^2 + 2kv}{4} = -\cfrac{(2k^2 - 2kv)}{4} + \cfrac{v^2}{4} \leq \cfrac{v^2}{4}$.

    Logo, $e \leq \cfrac{v^2}{4}$. \CQD

\end{document}
