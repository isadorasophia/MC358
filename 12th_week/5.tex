%=====================================================================
% jhwhw.cls
% Provide jhwhw.cls class
%=====================================================================

%=====================================================================
% Identification
%=====================================================================

\documentclass[12pt, letterpaper]{report}

\usepackage[utf8]{inputenc}

\usepackage{graphicx}
\usepackage{fancyhdr}
\usepackage[top=1in,bottom=1in,left=1in,right=1in]{geometry}
\usepackage{empheq}
\usepackage{ifthen}

\usepackage{enumitem}

\usepackage{amssymb}

\newcommand*{\CQD}{\hfill\ensuremath{\blacksquare}}%

%=====================================================================
% Commands
%=====================================================================

  \setlength{\headheight}{15pt}
  \lhead{Isadora S. G. Rodopoulos}\chead{MC358}\rhead{06 de Junho, 2016}
  \lfoot{}\cfoot{\thepage}\rfoot{}
  \pagestyle{fancy}

\ifx\pdfoutput\undefined                         %LaTeX
  \RequirePackage[ps2pdf,bookmarks=true]{hyperref}
  \hypersetup{ %
    pdfauthor   = {\@author},
    pdftitle    = {\@title},
    pdfcreator  = {LaTeX with hyperref package},
    pdfproducer = {dvips + ps2pdf}
  }
\else                                            %PDFLaTeX
  \RequirePackage[pdftex,bookmarks=true]{hyperref}
  \hypersetup{ %
    pdfauthor   = {\@author},
    pdftitle    = {\@title},
    pdfcreator  = {LaTeX with hyperref package},
    pdfproducer = {dvips + ps2pdf}
  }
\pdfadjustspacing=1
\fi

% Set up counters for problems and subsections

\newcounter{ProblemNum}
\newcounter{SubProblemNum}[ProblemNum]

\renewcommand{\theProblemNum}{\arabic{ProblemNum}}
\renewcommand{\theSubProblemNum}{\alph{SubProblemNum}}


\newcommand*{\anyproblem}[1]{\newpage\subsection*{#1}}
\newcommand*{\problem}[1]{\stepcounter{ProblemNum} %
   \anyproblem{Questão #1}}
\newcommand*{\soln}[1]{\subsubsection*{#1}}
\newcommand*{\solution}{\soln{Solução}}
\renewcommand*{\part}{\stepcounter{SubProblemNum} %
  \soln{Parte (\theSubProblemNum)}}

\renewcommand{\theenumi}{(\alph{enumi})}
\renewcommand{\labelenumi}{\theenumi}
\renewcommand{\theenumii}{\roman{enumii}}

% \def\problemmark{}

% % Typesetting problems

% % \newcommand*{\prob}[1]{\newpage \noindent \textbf{\Large #1}}
% % \newcommand*{\problem}[1]{\stepcounter{ProblemNum} \prob{Problem %
% % \theProblemNum.}}
% % \newcommand*{\soln}[1]{\\ \noindent \textbf{\Large #1}}
% % \newcommand*{\solution}{\soln{Solution}}
% % \renewcommand*{\part}{\\ \noindent \stepcounter{SubProblemNum} %
% % \textbf{\Large Part (\theSubProblemNum)}}

% \newcommand\problem{\@startsection{problem}{1}{\z@}%
%                      {-3.25ex \@plus -1ex \@minus -.2ex}%
%                      {1.5ex \@plus .2ex}%
%                      {\normalfont\large\bfseries}}

\begin{document}

\problem{5.}
Prove que para qualquer grafo $G = (V, E)$, vale a seguinte desigualdade

$$ |E| \ge |V| - c(G)$$

sendo que $c(G)$ denota o número de componentes conexas de $G$.

\solution
  Prova por indução em $n = $ número de vértices.

  \textbf{Base:} Em um grafo $G$ com $n = 0$ vértices, temos $0$ arestas e $c(G) = 0$ e $0 \geq 0$.
  Em um grafo $G$ com $n = 1$ vértices, temos, no mínimo, $v = 0$ e $v \geq 1 - 1$.

  \textbf{Hipótese:} Para qualquer grafo $G = (V, E)$, com $V = n \geq 0$, há a relação $ |E| \ge |V| - c(G) $.

  \textbf{Passo:} Suponhamos um grafo $G$ com $n + 1$ vértices. Se elimarmos um vértice e todas as suas arestas, estaremos dispostos
  de um grafo $G'$ com $n$ vértices e, P.H.I., $|E'| \geq n - c(G')$.

  Ao adicionarmos o vértice e suas arestas novamente, estaremos dispostos de dois casos:
  \begin{enumerate}[label=\arabic*.]
    \item Se o vértice não tinha nenhuma aresta, temos que $|E| = |E'|$ e $c(G) = c(G') + 1$.
    Ou seja, $|E'| \geq n - c(G') \rightarrow |E'| \geq n + 1 - (c(G') + 1) \rightarrow |E| \geq |V| - c(G)$.

    \item Se o vértice tinha, pelo menos, $x$ arestas, temos que $|E| = |E'| + x$ e $c(G) = c(G')$.
    Logo, $|E'| + x - 1 \geq n - c(G') \rightarrow |E'| + x \geq n + 1 - c(G') \rightarrow |E| \geq |V| - c(G)$, que é válido para todo $x \geq 1$.
  \end{enumerate}

\end{document}
