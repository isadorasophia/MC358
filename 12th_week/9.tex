%=====================================================================
% jhwhw.cls
% Provide jhwhw.cls class
%=====================================================================

%=====================================================================
% Identification
%=====================================================================

\documentclass[12pt, letterpaper]{report}

\usepackage[utf8]{inputenc}

\usepackage{graphicx}
\usepackage{fancyhdr}
\usepackage[top=1in,bottom=1in,left=1in,right=1in]{geometry}
\usepackage{empheq}
\usepackage{ifthen}

\usepackage{enumitem}

\usepackage{amssymb}
\usepackage{amsmath}
\newcommand*{\CQD}{\hfill\ensuremath{\blacksquare}}%


%=====================================================================
% Commands
%=====================================================================

  \setlength{\headheight}{15pt}
  \lhead{Isadora S. G. Rodopoulos}\chead{MC358}\rhead{06 de Junho, 2016}
  \lfoot{}\cfoot{\thepage}\rfoot{}
  \pagestyle{fancy}

\ifx\pdfoutput\undefined                         %LaTeX
  \RequirePackage[ps2pdf,bookmarks=true]{hyperref}
  \hypersetup{ %
    pdfauthor   = {\@author},
    pdftitle    = {\@title},
    pdfcreator  = {LaTeX with hyperref package},
    pdfproducer = {dvips + ps2pdf}
  }
\else                                            %PDFLaTeX
  \RequirePackage[pdftex,bookmarks=true]{hyperref}
  \hypersetup{ %
    pdfauthor   = {\@author},
    pdftitle    = {\@title},
    pdfcreator  = {LaTeX with hyperref package},
    pdfproducer = {dvips + ps2pdf}
  }
\pdfadjustspacing=1
\fi

% Set up counters for problems and subsections

\newcounter{ProblemNum}
\newcounter{SubProblemNum}[ProblemNum]

\renewcommand{\theProblemNum}{\arabic{ProblemNum}}
\renewcommand{\theSubProblemNum}{\alph{SubProblemNum}}


\newcommand*{\anyproblem}[1]{\newpage\subsection*{#1}}
\newcommand*{\problem}[1]{\stepcounter{ProblemNum} %
   \anyproblem{Questão #1}}
\newcommand*{\soln}[1]{\subsubsection*{#1}}
\newcommand*{\solution}{\soln{Solução}}
\renewcommand*{\part}{\stepcounter{SubProblemNum} %
  \soln{Parte (\theSubProblemNum)}}

\renewcommand{\theenumi}{(\alph{enumi})}
\renewcommand{\labelenumi}{\theenumi}
\renewcommand{\theenumii}{\roman{enumii}}

% \def\problemmark{}

% % Typesetting problems

% % \newcommand*{\prob}[1]{\newpage \noindent \textbf{\Large #1}}
% % \newcommand*{\problem}[1]{\stepcounter{ProblemNum} \prob{Problem %
% % \theProblemNum.}}
% % \newcommand*{\soln}[1]{\\ \noindent \textbf{\Large #1}}
% % \newcommand*{\solution}{\soln{Solution}}
% % \renewcommand*{\part}{\\ \noindent \stepcounter{SubProblemNum} %
% % \textbf{\Large Part (\theSubProblemNum)}}

% \newcommand\problem{\@startsection{problem}{1}{\z@}%
%                      {-3.25ex \@plus -1ex \@minus -.2ex}%
%                      {1.5ex \@plus .2ex}%
%                      {\normalfont\large\bfseries}}

\begin{document}

\problem{9.}
  Uma árvore é um grafo conexo sem ciclos. Prove, por indução forte no número de arestas, que toda árvore com pelo menos uma aresta tem pelo menos dois vértices de grau 1.

\solution
  Prova com indução forte em $n = $ número de arestas.

  \textbf{Base:} Se a árvore possui $ n = 1$ arestas, então ela possui, pelo menos, dois vértices de grau 1 aos quais a aresta está ligando (uma vez que ciclos não são permitidos).

  \textbf{Hipótese:} Toda árvore com $ n > k \geq 1$ arestas tem, pelo menos, dois vértices de grau um.

  \textbf{Passo:} Considere uma árvore com $n + 1$ arestas. Se dividi-la em dois componentes conexos através da remoção de uma aresta, estaremos dispostos de dois componentes conexos com $k$ e $n - k$ arestas.

  \begin{enumerate}[label=\arabic*.]
    \item Caso um dos componentes conexos não tenha nenhuma aresta, o vértice desse componente possuía grau 1 (antes da retirada da aresta). O componente conexo restante se trata de uma árvore com $n - k$ arestas e, P.H.I., possui dois vértices de grau 1. Ao recolocar a aresta removida, há duas possibilidades: 
      \begin{enumerate}[label=\arabic*.]
        \item O vértice da árvore que a aresta reuniu não se tratava de um vértice de grau 1, e a árvore de $n + 1$ arestas possui três ou mais vértices de grau 1.
        \item O vértice da árvore possuía grau 1 - entretanto, o vértice que a aresta reuniu já se trata de um vértice de grau 1. Logo, a árvore total de $n + 1$ arestas permanece pelo menos dois vértices de grau 1. \CQD
      \end{enumerate} 
    \item Caso ambos os componentes sejam árvores e P.H.I. possuem, pelo menos, dois vértices de grau 1. Ao juntá-las novamente, temos duas possibilidades:

      \begin{enumerate}[label=\arabic*.]
        \item Ambos os vértices que a aresta reuniu não se tratavam de um vértice de grau 1, e a árvore de $n + 1$ arestas possui quatro ou mais vértices de grau 1.
        \item Um dos vértices (ou ambos) se tratavam de um vértice de grau 1 - entretanto, cada uma das árvores possuía pelo menos mais um vértice de grau 1. Logo, a árvore total de $n + 1$ arestas possui pelo menos dois vértices de grau 1. \CQD
      \end{enumerate}
  \end{enumerate}

\end{document}
