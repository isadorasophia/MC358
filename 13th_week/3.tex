%=====================================================================
% jhwhw.cls
% Provide jhwhw.cls class
%=====================================================================

%=====================================================================
% Identification
%=====================================================================

\documentclass[12pt, letterpaper]{report}

\usepackage[utf8]{inputenc}

\usepackage{graphicx}
\usepackage{fancyhdr}
\usepackage[top=1in,bottom=1in,left=1in,right=1in]{geometry}
\usepackage{empheq}
\usepackage{ifthen}

\usepackage{enumitem}

\usepackage{mathrsfs}

\usepackage{amssymb}
\newcommand*{\CQD}{\hfill\ensuremath{\blacksquare}}%

%=====================================================================
% Commands
%=====================================================================

  \setlength{\headheight}{15pt}
  \lhead{Isadora S. G. Rodopoulos}\chead{MC358}\rhead{13 de Junho, 2016}
  \lfoot{}\cfoot{\thepage}\rfoot{}
  \pagestyle{fancy}

\ifx\pdfoutput\undefined                         %LaTeX
  \RequirePackage[ps2pdf,bookmarks=true]{hyperref}
  \hypersetup{ %
    pdfauthor   = {\@author},
    pdftitle    = {\@title},
    pdfcreator  = {LaTeX with hyperref package},
    pdfproducer = {dvips + ps2pdf}
  }
\else                                            %PDFLaTeX
  \RequirePackage[pdftex,bookmarks=true]{hyperref}
  \hypersetup{ %
    pdfauthor   = {\@author},
    pdftitle    = {\@title},
    pdfcreator  = {LaTeX with hyperref package},
    pdfproducer = {dvips + ps2pdf}
  }
\pdfadjustspacing=1
\fi

% Set up counters for problems and subsections

\newcounter{ProblemNum}
\newcounter{SubProblemNum}[ProblemNum]

\renewcommand{\theProblemNum}{\arabic{ProblemNum}}
\renewcommand{\theSubProblemNum}{\alph{SubProblemNum}}


\newcommand*{\anyproblem}[1]{\newpage\subsection*{#1}}
\newcommand*{\problem}[1]{\stepcounter{ProblemNum} %
   \anyproblem{Questão #1}}
\newcommand*{\soln}[1]{\subsubsection*{#1}}
\newcommand*{\solution}{\soln{Solução}}
\renewcommand*{\part}{\stepcounter{SubProblemNum} %
  \soln{Parte (\theSubProblemNum)}}

\renewcommand{\theenumi}{(\alph{enumi})}
\renewcommand{\labelenumi}{\theenumi}
\renewcommand{\theenumii}{\roman{enumii}}

% \def\problemmark{}

% % Typesetting problems

% % \newcommand*{\prob}[1]{\newpage \noindent \textbf{\Large #1}}
% % \newcommand*{\problem}[1]{\stepcounter{ProblemNum} \prob{Problem %
% % \theProblemNum.}}
% % \newcommand*{\soln}[1]{\\ \noindent \textbf{\Large #1}}
% % \newcommand*{\solution}{\soln{Solution}}
% % \renewcommand*{\part}{\\ \noindent \stepcounter{SubProblemNum} %
% % \textbf{\Large Part (\theSubProblemNum)}}

% \newcommand\problem{\@startsection{problem}{1}{\z@}%
%                      {-3.25ex \@plus -1ex \@minus -.2ex}%
%                      {1.5ex \@plus .2ex}%
%                      {\normalfont\large\bfseries}}

\begin{document}

\problem{3.}
  Seja $ G $ um grafo simples com $ k \ge 2 $ componentes conexas, cada uma das quais é um subgrafo Hamiltoniano de  $ G $. Encontre uma fórmula para a menor quantidade de arestas que devem ser inseridas em $ G $ apara que este se torne um grafo Hamiltoniano e prove sua corretude por indução em $ k $.

\solution
  A fórmula para a menor quantidade de arestas é $2k - 2$. Prova por indução em $k =$ componentes conexas.

  \textbf{Base:} Se $k = 2$, temos duas componentes conexas Hamiltonianas. Basta considerar dois vértices unidos por uma aresta em cada um dos subgrafos e denominá-los, arbitrariamente, de \textit{início} e \textit{fim}. Em seguinte, basta unir o vértice de início da primeira componente com o vértice final da segunda componente (e vice-versa).
  Acrescentando-se, assim, $2k - 2 = 2$ arestas.

  \textbf{Hipótese:} Em um grafo simples $G$ de $k > n \geq 2$ componentes conexas, em que cada uma delas é um subgrafo Hamiltoniano, é possível formar um grafo Hamiltoniano acrescentando-se $k$ arestas.

  \textbf{Passo:} Considere um grafo $G$ de $k$ componentes conexas. Separe-as em dois grupos, com $n$ e $k - n$ componentes conexas, cada um. P.H.I., cada um dos componentes necessita de $2n - 2$ e $2(k - n) - 2$ arestas para formar uma grafo Hamiltoniano.
  
  Estaremos dispostos de dois subgrafos Hamiltonianos, com $2n - 2 + 2k - 2n - 2 = 2k - 4$ arestas acrescentadas até então. Acrescentando mais $2 \cdot 2 - 2 = 2$ arestas, conclui-se a adição de $2k - 2$ arestas para formar um grafo Hamiltoniano. \CQD

\end{document}
