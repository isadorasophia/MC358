%=====================================================================
% jhwhw.cls
% Provide jhwhw.cls class
%=====================================================================

%=====================================================================
% Identification
%=====================================================================

\documentclass[12pt, letterpaper]{report}

\usepackage[utf8]{inputenc}

\usepackage{graphicx}
\usepackage{fancyhdr}
\usepackage[top=1in,bottom=1in,left=1in,right=1in]{geometry}
\usepackage{empheq}
\usepackage{ifthen}

\usepackage{enumitem}

\usepackage{amssymb}

\newcommand*{\CQD}{\hfill\ensuremath{\blacksquare}}%

%=====================================================================
% Commands
%=====================================================================

  \setlength{\headheight}{15pt}
  \lhead{Isadora S. G. Rodopoulos}\chead{MC358}\rhead{13 de Junho, 2016}
  \lfoot{}\cfoot{\thepage}\rfoot{}
  \pagestyle{fancy}

\ifx\pdfoutput\undefined                         %LaTeX
  \RequirePackage[ps2pdf,bookmarks=true]{hyperref}
  \hypersetup{ %
    pdfauthor   = {\@author},
    pdftitle    = {\@title},
    pdfcreator  = {LaTeX with hyperref package},
    pdfproducer = {dvips + ps2pdf}
  }
\else                                            %PDFLaTeX
  \RequirePackage[pdftex,bookmarks=true]{hyperref}
  \hypersetup{ %
    pdfauthor   = {\@author},
    pdftitle    = {\@title},
    pdfcreator  = {LaTeX with hyperref package},
    pdfproducer = {dvips + ps2pdf}
  }
\pdfadjustspacing=1
\fi

% Set up counters for problems and subsections

\newcounter{ProblemNum}
\newcounter{SubProblemNum}[ProblemNum]

\renewcommand{\theProblemNum}{\arabic{ProblemNum}}
\renewcommand{\theSubProblemNum}{\alph{SubProblemNum}}


\newcommand*{\anyproblem}[1]{\newpage\subsection*{#1}}
\newcommand*{\problem}[1]{\stepcounter{ProblemNum} %
   \anyproblem{Questão #1}}
\newcommand*{\soln}[1]{\subsubsection*{#1}}
\newcommand*{\solution}{\soln{Solução}}
\renewcommand*{\part}{\stepcounter{SubProblemNum} %
  \soln{Parte (\theSubProblemNum)}}

\renewcommand{\theenumi}{(\alph{enumi})}
\renewcommand{\labelenumi}{\theenumi}
\renewcommand{\theenumii}{\roman{enumii}}

% \def\problemmark{}

% % Typesetting problems

% % \newcommand*{\prob}[1]{\newpage \noindent \textbf{\Large #1}}
% % \newcommand*{\problem}[1]{\stepcounter{ProblemNum} \prob{Problem %
% % \theProblemNum.}}
% % \newcommand*{\soln}[1]{\\ \noindent \textbf{\Large #1}}
% % \newcommand*{\solution}{\soln{Solution}}
% % \renewcommand*{\part}{\\ \noindent \stepcounter{SubProblemNum} %
% % \textbf{\Large Part (\theSubProblemNum)}}

% \newcommand\problem{\@startsection{problem}{1}{\z@}%
%                      {-3.25ex \@plus -1ex \@minus -.2ex}%
%                      {1.5ex \@plus .2ex}%
%                      {\normalfont\large\bfseries}}

\begin{document}

\problem{5.}
  Dado um tabuleiro $ T_{n,m} $ de xadrez $ n \times m $, um percurso equino é uma sequência de casas visitadas por um cavalo (de xadrez) que passa por todas as casas do tabuleiro exatamente uma vez. Um circuito equino é um percurso equino no qual a primeira e a última casa coincidem.

  Há percurso equino no tabuleiro $ T_{4,4} $? Justifique.

  Apresente um argumento que prova que não há circuito equino no tabuleiro $ T_{4,4} $.

\solution
  Não. Suponha um tabuleiro $T_{4, 4}$ como a tabela a seguir:

  \begin{center}
    \begin{tabular}{ | c | c | c | c | }
      \hline
      1 & 2 & 3 & 4 \\ \hline
      5 & 6 & 7 & 8 \\ \hline
      9 & 10 & 11 & 12 \\ \hline
      13 & 14 & 15 & 16 \\
      \hline
    \end{tabular}
  \end{center}

  É possível relacionar o tabuleiro com um grafo $v =$ 16 vértices, em que cada vértice se liga com os vértices que representam possíveis movimentos do cavalo. O grafo está representado abaixo:

  \begin{center}
    \includegraphics[width=3in]{graph}
  \end{center}

  É possível visualizar que os ciclos com os vértices de grau 2 devem fazer parte do início das iterações e do fim (1, 7, 10, 16) e (4, 11, 6, 13) para que cada vértice sja percorrido uma única vez, os quais seriam as casas do canto.

  Entretanto, os restantes dos vértices de grau 3 não são possíveis de serem todos percorridos de forma a relacionar o caminho de início e fim. Por exemplo, um caminho a partir do ciclo (1, 10, 7, 16), que seria (10, 8), (8, 2), (2, 9) e (15, 6) for feito, os vértices 3 e 12 não são percorridos.

  Por outro lado, se o caminho for a partir do ciclo (4, 11, 6, 13), com (6, 12), (12, 3), (3, 5), (5, 15) e (15, 7), os vértices 8 e 2 não são percorridos.

  Logo, não há um percurso equino disponível no tabuleiro $T_{4, 4}$.
  \CQD
\end{document}
