%=====================================================================
% jhwhw.cls
% Provide jhwhw.cls class
%=====================================================================

%=====================================================================
% Identification
%=====================================================================

\documentclass[12pt, letterpaper]{report}

\usepackage[utf8]{inputenc}

\usepackage{graphicx}
\usepackage{fancyhdr}
\usepackage[top=1in,bottom=1in,left=1in,right=1in]{geometry}
\usepackage{empheq}
\usepackage{ifthen}

\usepackage{enumitem}

\usepackage{amssymb}
\newcommand*{\CQD}{\hfill\ensuremath{\blacksquare}}%


%=====================================================================
% Commands
%=====================================================================

  \setlength{\headheight}{15pt}
  \lhead{Isadora S. G. Rodopoulos}\chead{MC358}\rhead{13 de Junho, 2016}
  \lfoot{}\cfoot{\thepage}\rfoot{}
  \pagestyle{fancy}

\ifx\pdfoutput\undefined                         %LaTeX
  \RequirePackage[ps2pdf,bookmarks=true]{hyperref}
  \hypersetup{ %
    pdfauthor   = {\@author},
    pdftitle    = {\@title},
    pdfcreator  = {LaTeX with hyperref package},
    pdfproducer = {dvips + ps2pdf}
  }
\else                                            %PDFLaTeX
  \RequirePackage[pdftex,bookmarks=true]{hyperref}
  \hypersetup{ %
    pdfauthor   = {\@author},
    pdftitle    = {\@title},
    pdfcreator  = {LaTeX with hyperref package},
    pdfproducer = {dvips + ps2pdf}
  }
\pdfadjustspacing=1
\fi

% Set up counters for problems and subsections

\newcounter{ProblemNum}
\newcounter{SubProblemNum}[ProblemNum]

\renewcommand{\theProblemNum}{\arabic{ProblemNum}}
\renewcommand{\theSubProblemNum}{\alph{SubProblemNum}}


\newcommand*{\anyproblem}[1]{\newpage\subsection*{#1}}
\newcommand*{\problem}[1]{\stepcounter{ProblemNum} %
   \anyproblem{Questão #1}}
\newcommand*{\soln}[1]{\subsubsection*{#1}}
\newcommand*{\solution}{\soln{Solução}}
\renewcommand*{\part}{\stepcounter{SubProblemNum} %
  \soln{Parte (\theSubProblemNum)}}

\renewcommand{\theenumi}{(\alph{enumi})}
\renewcommand{\labelenumi}{\theenumi}
\renewcommand{\theenumii}{\roman{enumii}}

% \def\problemmark{}

% % Typesetting problems

% % \newcommand*{\prob}[1]{\newpage \noindent \textbf{\Large #1}}
% % \newcommand*{\problem}[1]{\stepcounter{ProblemNum} \prob{Problem %
% % \theProblemNum.}}
% % \newcommand*{\soln}[1]{\\ \noindent \textbf{\Large #1}}
% % \newcommand*{\solution}{\soln{Solution}}
% % \renewcommand*{\part}{\\ \noindent \stepcounter{SubProblemNum} %
% % \textbf{\Large Part (\theSubProblemNum)}}

% \newcommand\problem{\@startsection{problem}{1}{\z@}%
%                      {-3.25ex \@plus -1ex \@minus -.2ex}%
%                      {1.5ex \@plus .2ex}%
%                      {\normalfont\large\bfseries}}

\begin{document}

\problem{8.}
  Dado um tabuleiro $ T_{n,m} $ de xadrez $ n \times m $, um percurso equino é uma sequência de casas visitadas por um cavalo (de xadrez) que passa por todas as casas do tabuleiro exatamente uma vez. Um circuito equino é um percurso equino no qual a primeira e a última casa coincidem.

  Prove que o grafo que representa os movimentos válidos de um cavalo no tabuleiro $ T_{n,m} $ onde para $ n $, $ m $ naturais positivos é bipartido.

\solution
  O grafo que representa os movimentos do cavalo de trata de um grafo $G$ com $v = nm$ vértices, em que cada aresta representaria uma possibilidade de movimento.
  Considerando um tabuleiro xadrez de $n \geq i \geq 1$ linhas e $m \geq j \geq 1$ colunas, é possível visualizar que as casas \textbf{pretas} representam $i + j$ par, enquanto as casas \textbf{brancas} representam $i + j$ ímpar.

  O cavalo sempre anda $+ 2 + 1 = 3$ ou $- 2 - 1= -3$ ou $-2 + 1 = -1$ ou $2 - 1 = 1$. De qualquer forma, a soma da linha e da coluna é um valor ímpar, o que significa que a cor sempre muda (pois ímpar com ímpar = par e par com ímpar = ímpar).

  Ou seja, o grafo do tabuleiro pode ser separado em grupos $V_1$ e $V_2$, em que um vértice $v \in V_1$ quando ele é preto e $v \in V_2$ quando ele é branco, por exemplo, constituindo um grafo bipartido.

\end{document}
