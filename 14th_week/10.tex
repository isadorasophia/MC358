%=====================================================================
% jhwhw.cls
% Provide jhwhw.cls class
%=====================================================================

%=====================================================================
% Identification
%=====================================================================

\documentclass[12pt, letterpaper]{report}

\usepackage[utf8]{inputenc}

\usepackage{graphicx}
\usepackage{fancyhdr}
\usepackage[top=1in,bottom=1in,left=1in,right=1in]{geometry}
\usepackage{empheq}
\usepackage{ifthen}

\usepackage{enumitem}

\usepackage{amssymb}
\usepackage{amsmath}
\newcommand*{\CQD}{\hfill\ensuremath{\blacksquare}}%


%=====================================================================
% Commands
%=====================================================================

  \setlength{\headheight}{15pt}
  \lhead{Isadora S. G. Rodopoulos}\chead{MC358}\rhead{17 de Junho, 2016}
  \lfoot{}\cfoot{\thepage}\rfoot{}
  \pagestyle{fancy}

\ifx\pdfoutput\undefined                         %LaTeX
  \RequirePackage[ps2pdf,bookmarks=true]{hyperref}
  \hypersetup{ %
    pdfauthor   = {\@author},
    pdftitle    = {\@title},
    pdfcreator  = {LaTeX with hyperref package},
    pdfproducer = {dvips + ps2pdf}
  }
\else                                            %PDFLaTeX
  \RequirePackage[pdftex,bookmarks=true]{hyperref}
  \hypersetup{ %
    pdfauthor   = {\@author},
    pdftitle    = {\@title},
    pdfcreator  = {LaTeX with hyperref package},
    pdfproducer = {dvips + ps2pdf}
  }
\pdfadjustspacing=1
\fi

% Set up counters for problems and subsections

\newcounter{ProblemNum}
\newcounter{SubProblemNum}[ProblemNum]

\renewcommand{\theProblemNum}{\arabic{ProblemNum}}
\renewcommand{\theSubProblemNum}{\alph{SubProblemNum}}


\newcommand*{\anyproblem}[1]{\newpage\subsection*{#1}}
\newcommand*{\problem}[1]{\stepcounter{ProblemNum} %
   \anyproblem{Questão #1}}
\newcommand*{\soln}[1]{\subsubsection*{#1}}
\newcommand*{\solution}{\soln{Solução}}
\renewcommand*{\part}{\stepcounter{SubProblemNum} %
  \soln{Parte (\theSubProblemNum)}}

\renewcommand{\theenumi}{(\alph{enumi})}
\renewcommand{\labelenumi}{\theenumi}
\renewcommand{\theenumii}{\roman{enumii}}

% \def\problemmark{}

% % Typesetting problems

% % \newcommand*{\prob}[1]{\newpage \noindent \textbf{\Large #1}}
% % \newcommand*{\problem}[1]{\stepcounter{ProblemNum} \prob{Problem %
% % \theProblemNum.}}
% % \newcommand*{\soln}[1]{\\ \noindent \textbf{\Large #1}}
% % \newcommand*{\solution}{\soln{Solution}}
% % \renewcommand*{\part}{\\ \noindent \stepcounter{SubProblemNum} %
% % \textbf{\Large Part (\theSubProblemNum)}}

% \newcommand\problem{\@startsection{problem}{1}{\z@}%
%                      {-3.25ex \@plus -1ex \@minus -.2ex}%
%                      {1.5ex \@plus .2ex}%
%                      {\normalfont\large\bfseries}}

\begin{document}

\problem{10.}
  Use indução para provar que qualquer grafo conexo com pelo menos dois vértices tem algum vértice cuja remoção (junto com as arestas que incidem nele) mantém o grafo conexo. Explicite o parâmetro usado na indução e se é uma indução fraca ou forte.

\solution
  Prova por indução em $n =$ número de vértices.

  \textbf{Base:} Se o grafo $G$ possui $n = 2$, a remoção de qualquer um dos vértices (e as arestas que incidem nele) mantém o grafo conexo, uma vez que haverá apenas um vértice.

  \textbf{Passo:} Um grafo $G$ com $n > k \geq 2$ vértices tem algum vértice cuja remoção (junto com as arestas que incidem nele) mantém o grafo conexo.

  \textbf{Hipótese:} Suponha um grafo $G$ com $n$ vértices. Remove um vértice arbitrário $v$, juntamente com as arestas que incidem nele. Estamos dispostos de duas possibilidades:

  \begin{enumerate}
    \item O grafo de $n - 1$ vértices continua conexo e o vértice $v$ satisfaz a hipótese para $G$. \CQD
    \item O grafo tornou-se desconexo e há dois componentes conexos de $k$ e $n - 1 - k$ vértices. P.H.I., em ambos os componentes há um vértice $v_1$ e $v_2$ cuja remoção mantém o componente conexo. 

    Retorne o vértice $v$ ao grafo. Assim, a remoção do vértice $v_1$ ou $v_2$ manterá o grafo conexo. \CQD
  \end{enumerate}
\end{document}
