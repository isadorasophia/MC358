%=====================================================================
% jhwhw.cls
% Provide jhwhw.cls class
%=====================================================================

%=====================================================================
% Identification
%=====================================================================

\documentclass[12pt, letterpaper]{report}

\usepackage[utf8]{inputenc}

\usepackage{graphicx}
\usepackage{fancyhdr}
\usepackage[top=1in,bottom=1in,left=1in,right=1in]{geometry}
\usepackage{empheq}
\usepackage{ifthen}

\usepackage{enumitem}

\usepackage{mathrsfs}

\usepackage{amssymb}
\newcommand*{\CQD}{\hfill\ensuremath{\blacksquare}}%

%=====================================================================
% Commands
%=====================================================================

  \setlength{\headheight}{15pt}
  \lhead{Isadora S. G. Rodopoulos}\chead{MC358}\rhead{17 de Junho, 2016}
  \lfoot{}\cfoot{\thepage}\rfoot{}
  \pagestyle{fancy}

\ifx\pdfoutput\undefined                         %LaTeX
  \RequirePackage[ps2pdf,bookmarks=true]{hyperref}
  \hypersetup{ %
    pdfauthor   = {\@author},
    pdftitle    = {\@title},
    pdfcreator  = {LaTeX with hyperref package},
    pdfproducer = {dvips + ps2pdf}
  }
\else                                            %PDFLaTeX
  \RequirePackage[pdftex,bookmarks=true]{hyperref}
  \hypersetup{ %
    pdfauthor   = {\@author},
    pdftitle    = {\@title},
    pdfcreator  = {LaTeX with hyperref package},
    pdfproducer = {dvips + ps2pdf}
  }
\pdfadjustspacing=1
\fi

% Set up counters for problems and subsections

\newcounter{ProblemNum}
\newcounter{SubProblemNum}[ProblemNum]

\renewcommand{\theProblemNum}{\arabic{ProblemNum}}
\renewcommand{\theSubProblemNum}{\alph{SubProblemNum}}


\newcommand*{\anyproblem}[1]{\newpage\subsection*{#1}}
\newcommand*{\problem}[1]{\stepcounter{ProblemNum} %
   \anyproblem{Questão #1}}
\newcommand*{\soln}[1]{\subsubsection*{#1}}
\newcommand*{\solution}{\soln{Solução}}
\renewcommand*{\part}{\stepcounter{SubProblemNum} %
  \soln{Parte (\theSubProblemNum)}}

\renewcommand{\theenumi}{(\alph{enumi})}
\renewcommand{\labelenumi}{\theenumi}
\renewcommand{\theenumii}{\roman{enumii}}

% \def\problemmark{}

% % Typesetting problems

% % \newcommand*{\prob}[1]{\newpage \noindent \textbf{\Large #1}}
% % \newcommand*{\problem}[1]{\stepcounter{ProblemNum} \prob{Problem %
% % \theProblemNum.}}
% % \newcommand*{\soln}[1]{\\ \noindent \textbf{\Large #1}}
% % \newcommand*{\solution}{\soln{Solution}}
% % \renewcommand*{\part}{\\ \noindent \stepcounter{SubProblemNum} %
% % \textbf{\Large Part (\theSubProblemNum)}}

% \newcommand\problem{\@startsection{problem}{1}{\z@}%
%                      {-3.25ex \@plus -1ex \@minus -.2ex}%
%                      {1.5ex \@plus .2ex}%
%                      {\normalfont\large\bfseries}}

\begin{document}

\problem{2.}
  Um grafo não dirigido $ G $ possui uma largura $ w $ se os seus vértices podem ser dispostos em uma sequência $ v_1, v_2, v_3, \dots, v_n $, tal que cada vértice $ v_i $ é vizinho de no máximo $ w $ dos vértices que o precede (vértice $ v_j $ precede $ v_i $ se $ j < i $). Prove por indução que todo grafo com largura $ w $ é $ (w + 1) $-colorível.

\solution
  Prova por indução em $w =$ largura do grafo $G$.

  \textbf{Base:} Se $w = 1$ em um grafo $G$ não dirigido, disposto em uma sequência $v_1, v_2, v_3, \dots, v_n$, todo vértice $v_i$ possui, no máximo, $1$ vizinho que o procede. Ou seja, basta colorir todo $v_j$ que procede um vértice $v_i$ com a cor \textbf{A} e \textbf{B}, respectivamente. Ou seja, $G$ é $ (w + 1) = 2$ colorível.

  \textbf{Hipótese:} Um grafo com largura $w > 1$ é $(w + 1)$ colorível.

  \textbf{Passo:} Considere um grafo $G$ com largura $w + 1$. Em seguida, elimine um precedente de cada vértice com $w + 1$ vizinhos precedentes, removendo a aresta entre $v_j$ e $v_i$.

  O grafo está disposto de $v_n$ vértices de largura $w$. P.H.I., este grafo é $w + 1$ colorível.

  Em seguida, restaure os precedentes, adicionando novamente as arestas removidas. Para colorir corretamente o gráfico, em cada $v_j$ e $v_i$ de aresta restaurada com mesma cor, basta colorir um dos vértices de uma nova cor. Assim, conclui-se que o grafo é $w + 1$ colorível.

\end{document}