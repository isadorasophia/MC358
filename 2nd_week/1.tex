%=====================================================================
% jhwhw.cls
% Provide jhwhw.cls class
%=====================================================================

%=====================================================================
% Identification
%=====================================================================

\documentclass[12pt, letterpaper]{report}

\usepackage[utf8]{inputenc}

\usepackage{graphicx}
\usepackage{fancyhdr}
\usepackage[top=1in,bottom=1in,left=1in,right=1in]{geometry}
\usepackage{empheq}
\usepackage{ifthen}

\usepackage{enumitem}

%=====================================================================
% Commands
%=====================================================================

  \setlength{\headheight}{15pt}
  \lhead{Isadora S. G. Rodopoulos}\chead{MC358}\rhead{15 de Março, 2016}
  \lfoot{}\cfoot{\thepage}\rfoot{}
  \pagestyle{fancy}

\ifx\pdfoutput\undefined                         %LaTeX
  \RequirePackage[ps2pdf,bookmarks=true]{hyperref}
  \hypersetup{ %
    pdfauthor   = {\@author},
    pdftitle    = {\@title},
    pdfcreator  = {LaTeX with hyperref package},
    pdfproducer = {dvips + ps2pdf}
  }
\else                                            %PDFLaTeX
  \RequirePackage[pdftex,bookmarks=true]{hyperref}
  \hypersetup{ %
    pdfauthor   = {\@author},
    pdftitle    = {\@title},
    pdfcreator  = {LaTeX with hyperref package},
    pdfproducer = {dvips + ps2pdf}
  }
\pdfadjustspacing=1
\fi

% Set up counters for problems and subsections

\newcounter{ProblemNum}
\newcounter{SubProblemNum}[ProblemNum]

\renewcommand{\theProblemNum}{\arabic{ProblemNum}}
\renewcommand{\theSubProblemNum}{\alph{SubProblemNum}}


\newcommand*{\anyproblem}[1]{\newpage\subsection*{#1}}
\newcommand*{\problem}[1]{\stepcounter{ProblemNum} %
   \anyproblem{Questão #1}}
\newcommand*{\soln}[1]{\subsubsection*{#1}}
\newcommand*{\solution}{\soln{Solução}}
\renewcommand*{\part}{\stepcounter{SubProblemNum} %
  \soln{Parte (\theSubProblemNum)}}

\renewcommand{\theenumi}{(\alph{enumi})}
\renewcommand{\labelenumi}{\theenumi}
\renewcommand{\theenumii}{\roman{enumii}}

% \def\problemmark{}

% % Typesetting problems

% % \newcommand*{\prob}[1]{\newpage \noindent \textbf{\Large #1}}
% % \newcommand*{\problem}[1]{\stepcounter{ProblemNum} \prob{Problem %
% % \theProblemNum.}}
% % \newcommand*{\soln}[1]{\\ \noindent \textbf{\Large #1}}
% % \newcommand*{\solution}{\soln{Solution}}
% % \renewcommand*{\part}{\\ \noindent \stepcounter{SubProblemNum} %
% % \textbf{\Large Part (\theSubProblemNum)}}

% \newcommand\problem{\@startsection{problem}{1}{\z@}%
%                      {-3.25ex \@plus -1ex \@minus -.2ex}%
%                      {1.5ex \@plus .2ex}%
%                      {\normalfont\large\bfseries}}

\begin{document}

\problem{1.}
Construa tabelas verdade para cada uma das seguintes expressões e diga se são tautologias, contradições ou nenhuma das duas.
  \begin{enumerate}[label=\arabic*.]
      \item $ p \lor (q \lor \lnot p) $
      \item $ p \land \lnot (q \lor \lnot q) $ 
      \item $ p \lor \lnot (q \lor \lnot q) $
  \end{enumerate}

\solution
  \begin{enumerate}[label=\arabic*.]
    \item Trata-se de uma tautologia - todas as condições estabelecem uma verdade.
    \begin{table}[ht!]
      \centering
      \begin{tabular}{| c | c | c | c |}
        \hline
        $ p $ & $ q $ & $ q \lor \lnot p $ & $ p \lor (q \lor \lnot p) $ \\
        \hline
        F & F & V & V \\ \hline
        F & V & V & V \\ \hline
        V & F & F & V \\ \hline
        V & V & V & V \\ \hline
      \end{tabular}
    \end{table}

    \item Trata-se de uma contradição, uma vez que todas as condições são falsas.
    \begin{table}[ht!]
      \centering
      \begin{tabular}{| c | c | c | c |}
        \hline
        $ p $ & $ q $ & $ \lnot(q \lor \lnot q) $ & $ p \land \lnot(q \lor \lnot q) $ \\
        \hline
        F & F & F & F \\ \hline
        F & V & F & F \\ \hline
        V & F & F & F \\ \hline
        V & V & F & F \\ \hline
      \end{tabular}
    \end{table}

    \item Não se trata de uma tautologia tampouco de uma contradição, dado que os resultados são adversos em relação às diferentes condições.
    \begin{table}[ht!]
      \centering
      \begin{tabular}{| c | c | c | c |}
        \hline
        $ p $ & $ q $ & $ \lnot(q \lor \lnot q) $ & $ p \lor \lnot(q \lor \lnot q) $ \\
        \hline
        F & F & F & F \\ \hline
        F & V & F & F \\ \hline
        V & F & F & V \\ \hline
        V & V & F & V \\ \hline
      \end{tabular}
    \end{table}
  \end{enumerate}

\end{document}