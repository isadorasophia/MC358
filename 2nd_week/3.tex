%=====================================================================
% jhwhw.cls
% Provide jhwhw.cls class
%=====================================================================

%=====================================================================
% Identification
%=====================================================================

\documentclass[12pt, letterpaper]{report}

\usepackage[utf8]{inputenc}

\usepackage{graphicx}
\usepackage{fancyhdr}
\usepackage[top=1in,bottom=1in,left=1in,right=1in]{geometry}
\usepackage{empheq}
\usepackage{ifthen}

\usepackage{enumitem}

%=====================================================================
% Commands
%=====================================================================

  \setlength{\headheight}{15pt}
  \lhead{Isadora S. G. Rodopoulos}\chead{MC358}\rhead{15 de Março, 2016}
  \lfoot{}\cfoot{\thepage}\rfoot{}
  \pagestyle{fancy}

\ifx\pdfoutput\undefined                         %LaTeX
  \RequirePackage[ps2pdf,bookmarks=true]{hyperref}
  \hypersetup{ %
    pdfauthor   = {\@author},
    pdftitle    = {\@title},
    pdfcreator  = {LaTeX with hyperref package},
    pdfproducer = {dvips + ps2pdf}
  }
\else                                            %PDFLaTeX
  \RequirePackage[pdftex,bookmarks=true]{hyperref}
  \hypersetup{ %
    pdfauthor   = {\@author},
    pdftitle    = {\@title},
    pdfcreator  = {LaTeX with hyperref package},
    pdfproducer = {dvips + ps2pdf}
  }
\pdfadjustspacing=1
\fi

% Set up counters for problems and subsections

\newcounter{ProblemNum}
\newcounter{SubProblemNum}[ProblemNum]

\renewcommand{\theProblemNum}{\arabic{ProblemNum}}
\renewcommand{\theSubProblemNum}{\alph{SubProblemNum}}


\newcommand*{\anyproblem}[1]{\newpage\subsection*{#1}}
\newcommand*{\problem}[1]{\stepcounter{ProblemNum} %
   \anyproblem{Questão #1}}
\newcommand*{\soln}[1]{\subsubsection*{#1}}
\newcommand*{\solution}{\soln{Solução}}
\renewcommand*{\part}{\stepcounter{SubProblemNum} %
  \soln{Parte (\theSubProblemNum)}}

\renewcommand{\theenumi}{(\alph{enumi})}
\renewcommand{\labelenumi}{\theenumi}
\renewcommand{\theenumii}{\roman{enumii}}

% \def\problemmark{}

% % Typesetting problems

% % \newcommand*{\prob}[1]{\newpage \noindent \textbf{\Large #1}}
% % \newcommand*{\problem}[1]{\stepcounter{ProblemNum} \prob{Problem %
% % \theProblemNum.}}
% % \newcommand*{\soln}[1]{\\ \noindent \textbf{\Large #1}}
% % \newcommand*{\solution}{\soln{Solution}}
% % \renewcommand*{\part}{\\ \noindent \stepcounter{SubProblemNum} %
% % \textbf{\Large Part (\theSubProblemNum)}}

% \newcommand\problem{\@startsection{problem}{1}{\z@}%
%                      {-3.25ex \@plus -1ex \@minus -.2ex}%
%                      {1.5ex \@plus .2ex}%
%                      {\normalfont\large\bfseries}}

\begin{document}

\problem{3.}
Em cada um dos casos abaixo, determine se as proposições são logicamente equivalentes. Justifique.
  \begin{enumerate}[label=\arabic*.]
      \item $ ((\forall x \in A)P(x)) \land ((\forall x \in B)P(x))$ e $((\forall x \in A \cup B)P(x)) $ 
      \item $ ((\exists x \in A)P(x)) \lor ((\exists x \in B)Q(x))$ e $((\exists x \in A \cup B)P(x) \lor Q(x)) $
      \item $ ((\forall x \in A)P(x)) \lor ((\forall x \in B)P(x))$ e $((\forall x \in A \cup B)P(x)) $
      \item $ ((\exists x \in A)P(x)) \land ((\exists x \in B)Q(x))$ e $((\exists x \in A \cup B)P(x) \lor Q(x)) $
  \end{enumerate}

\solution
  \begin{enumerate}[label=\arabic*.]
      \item São logicamente equivalentes. 

      Partindo da primeira proposição, $(\forall x \in A) P(x)$ = $(\forall x \in U) (x \in A) \rightarrow P(x)$, ou seja, ambas as expressões são $(\forall x \in U) ((x \in A) \rightarrow P(x)) \land ((x \in B) \rightarrow P(x))$ que é equivalente a 
      $(\forall x \in U) ((x \in A) \lor (x \in B)) \rightarrow P(x))$ e corresponde ao conjunto $\forall x \in A \cup B$. Assim, garante-se a equivalência lógica entre as proposições.

      % São logicamente equivalentes. Isto é, a primeira proposição afirma que $\forall x \in A$ condiz $P(x)$ \textbf{e} $\forall x \in B$ condiz $P(x)$. A segunda proposição consta que todos os elementos pertencentes a $A$ \textbf{ou} $B$ dizem respeito a $P(x)$ - o que se pode inferir a partir da primeira proposição, garantindo a equivalência.

      \item As proposições não são logicamente equivalentes. Se $ \exists x \in A $ que condiz $P(x)$ \textbf{ou} $\exists x \in B$ que $Q(x)$, então o conjunto união de todos os elementos de A \textbf{ou} B condiz a existência de um elemento $x$ tal que $P(x)$ ou $Q(x)$. 

      Entretanto, o caminho contrário não é equivalente pois, a partir da segunda proposição, um elemento de $A$ pode dizer respeito a $Q(x)$ e um elemento de $B$ diz respeito a $P(x)$: nesta condição, a primeira proposição nunca será correta. 

      \item Não são logicamente equivalentes. 

      Em uma resolução semelhante ao item 1, partirmos da primeira proposição, em que $(\forall x \in A) P(x)$ = $(\forall x \in U) (x \in A) \rightarrow P(x)$, ou seja, ambas as expressões são $(\forall x \in U) ((x \in A) \rightarrow P(x)) \lor ((x \in B) \rightarrow P(x))$ que é equivalente a $(\forall x \in U) ((x \in A) \land (x \in B)) \rightarrow P(x))$, o qual corresponde ao conjunto $\forall x \in A \cap B$. Isto é, a segunda proposição apenas se verificaria na interseção dos conjuntos.

      % A primeira proposição afirma que $\forall x \in A$ condiz $P(x)$ \textbf{ou} $\forall x \in B$ condiz $P(x)$. Ou seja, todos os elementos de $A$ \textbf{ou} $B$ dizem respeito a $P(x)$ - portanto, $(\forall x \in A \cup B)P(x)$.

      \item Não são logicamente equivalentes. Seguindo a mesma lógica do item 2, o caminho da segunda proposição para a primeira pode levar a uma falsidade, impossibilitando a equivalência.
  \end{enumerate}

\end{document}
