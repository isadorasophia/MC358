%=====================================================================
% jhwhw.cls
% Provide jhwhw.cls class
%=====================================================================

%=====================================================================
% Identification
%=====================================================================

\documentclass[12pt, letterpaper]{report}

\usepackage[utf8]{inputenc}

\usepackage{graphicx}
\usepackage{fancyhdr}
\usepackage[top=1in,bottom=1in,left=1in,right=1in]{geometry}
\usepackage{empheq}
\usepackage{ifthen}

\usepackage{enumitem}

%=====================================================================
% Commands
%=====================================================================

  \setlength{\headheight}{15pt}
  \lhead{Isadora S. G. Rodopoulos}\chead{MC358}\rhead{15 de Março, 2016}
  \lfoot{}\cfoot{\thepage}\rfoot{}
  \pagestyle{fancy}

\ifx\pdfoutput\undefined                         %LaTeX
  \RequirePackage[ps2pdf,bookmarks=true]{hyperref}
  \hypersetup{ %
    pdfauthor   = {\@author},
    pdftitle    = {\@title},
    pdfcreator  = {LaTeX with hyperref package},
    pdfproducer = {dvips + ps2pdf}
  }
\else                                            %PDFLaTeX
  \RequirePackage[pdftex,bookmarks=true]{hyperref}
  \hypersetup{ %
    pdfauthor   = {\@author},
    pdftitle    = {\@title},
    pdfcreator  = {LaTeX with hyperref package},
    pdfproducer = {dvips + ps2pdf}
  }
\pdfadjustspacing=1
\fi

% Set up counters for problems and subsections

\newcounter{ProblemNum}
\newcounter{SubProblemNum}[ProblemNum]

\renewcommand{\theProblemNum}{\arabic{ProblemNum}}
\renewcommand{\theSubProblemNum}{\alph{SubProblemNum}}


\newcommand*{\anyproblem}[1]{\newpage\subsection*{#1}}
\newcommand*{\problem}[1]{\stepcounter{ProblemNum} %
   \anyproblem{Questão #1}}
\newcommand*{\soln}[1]{\subsubsection*{#1}}
\newcommand*{\solution}{\soln{Solução}}
\renewcommand*{\part}{\stepcounter{SubProblemNum} %
  \soln{Parte (\theSubProblemNum)}}

\renewcommand{\theenumi}{(\alph{enumi})}
\renewcommand{\labelenumi}{\theenumi}
\renewcommand{\theenumii}{\roman{enumii}}

% \def\problemmark{}

% % Typesetting problems

% % \newcommand*{\prob}[1]{\newpage \noindent \textbf{\Large #1}}
% % \newcommand*{\problem}[1]{\stepcounter{ProblemNum} \prob{Problem %
% % \theProblemNum.}}
% % \newcommand*{\soln}[1]{\\ \noindent \textbf{\Large #1}}
% % \newcommand*{\solution}{\soln{Solution}}
% % \renewcommand*{\part}{\\ \noindent \stepcounter{SubProblemNum} %
% % \textbf{\Large Part (\theSubProblemNum)}}

% \newcommand\problem{\@startsection{problem}{1}{\z@}%
%                      {-3.25ex \@plus -1ex \@minus -.2ex}%
%                      {1.5ex \@plus .2ex}%
%                      {\normalfont\large\bfseries}}

\begin{document}

\problem{1.}
Um número real $ x $ é chamado de limite superior de um conjunto $ S $ de números reais se $ x $ for maior ou igual a todo membro de $ S $. Um número real $ x $ é um supremo de $ S $ se ele for um limite superior de $ S $ e for menor ou igual a todo limite superior de $ S $. Se $ S $ possuir um supremo, ele é único.
  \begin{enumerate}[label=\arabic*.]
    \item Usando quantificadores, expresse o fato de que $ x $ é um limite superior de $ S $.
    \item Usando quantificadores, expresse o fato de que $ x $ é o supremo de $ S $.
    \item Usando quantificadores, expresse o fato de que se $ x $ for um supremo de $ S $, então $ x $ é único (você pode representar os resultados dos dois itens anteriores com predicados).
  \end{enumerate}

\solution
  \part
    Considerando $L(x, s) = x \geq s$
  \part
  \begin{enumerate}[label=\arabic*.]
    \item $\forall s \in S \exists x(L(x, s))$
    \item $\forall y \forall s \in S(L(y, s)) \land \exists x (L(x, s) \land L(y, x))$
    \item $\forall y \forall s \in S(L(y, s)) \land \exists x (L(x, s) \land L(y, x)) \land (\forall w \not = x \rightarrow \lnot ((L(w, s) \land L(y, w))$
  \end{enumerate}

\end{document}