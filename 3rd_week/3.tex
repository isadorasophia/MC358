%=====================================================================
% jhwhw.cls
% Provide jhwhw.cls class
%=====================================================================

%=====================================================================
% Identification
%=====================================================================

\documentclass[12pt, letterpaper]{report}

\usepackage[utf8]{inputenc}

\usepackage{graphicx}
\usepackage{fancyhdr}
\usepackage[top=1in,bottom=1in,left=1in,right=1in]{geometry}
\usepackage{empheq}
\usepackage{ifthen}

\usepackage{enumitem}

%=====================================================================
% Commands
%=====================================================================

  \setlength{\headheight}{15pt}
  \lhead{Isadora S. G. Rodopoulos}\chead{MC358}\rhead{15 de Março, 2016}
  \lfoot{}\cfoot{\thepage}\rfoot{}
  \pagestyle{fancy}

\ifx\pdfoutput\undefined                         %LaTeX
  \RequirePackage[ps2pdf,bookmarks=true]{hyperref}
  \hypersetup{ %
    pdfauthor   = {\@author},
    pdftitle    = {\@title},
    pdfcreator  = {LaTeX with hyperref package},
    pdfproducer = {dvips + ps2pdf}
  }
\else                                            %PDFLaTeX
  \RequirePackage[pdftex,bookmarks=true]{hyperref}
  \hypersetup{ %
    pdfauthor   = {\@author},
    pdftitle    = {\@title},
    pdfcreator  = {LaTeX with hyperref package},
    pdfproducer = {dvips + ps2pdf}
  }
\pdfadjustspacing=1
\fi

% Set up counters for problems and subsections

\newcounter{ProblemNum}
\newcounter{SubProblemNum}[ProblemNum]

\renewcommand{\theProblemNum}{\arabic{ProblemNum}}
\renewcommand{\theSubProblemNum}{\alph{SubProblemNum}}


\newcommand*{\anyproblem}[1]{\newpage\subsection*{#1}}
\newcommand*{\problem}[1]{\stepcounter{ProblemNum} %
   \anyproblem{Questão #1}}
\newcommand*{\soln}[1]{\subsubsection*{#1}}
\newcommand*{\solution}{\soln{Solução}}
\renewcommand*{\part}{\stepcounter{SubProblemNum} %
  \soln{Parte (\theSubProblemNum)}}

\renewcommand{\theenumi}{(\alph{enumi})}
\renewcommand{\labelenumi}{\theenumi}
\renewcommand{\theenumii}{\roman{enumii}}

% \def\problemmark{}

% % Typesetting problems

% % \newcommand*{\prob}[1]{\newpage \noindent \textbf{\Large #1}}
% % \newcommand*{\problem}[1]{\stepcounter{ProblemNum} \prob{Problem %
% % \theProblemNum.}}
% % \newcommand*{\soln}[1]{\\ \noindent \textbf{\Large #1}}
% % \newcommand*{\solution}{\soln{Solution}}
% % \renewcommand*{\part}{\\ \noindent \stepcounter{SubProblemNum} %
% % \textbf{\Large Part (\theSubProblemNum)}}

% \newcommand\problem{\@startsection{problem}{1}{\z@}%
%                      {-3.25ex \@plus -1ex \@minus -.2ex}%
%                      {1.5ex \@plus .2ex}%
%                      {\normalfont\large\bfseries}}

\begin{document}

\problem{3.}
Demostre que para quaisquer conjuntos $A, B, C$ e $D$, as seguintes afirmações são
 verdadeiras:

  \begin{enumerate}[label=\arabic*.]
    \item Se $ x \in A, (A - B) \subset (C \cap D) $ e $ x \not \in D $, então $ x \in B $.
    \item Se $ x \in C $ e $ (A \cap C) \subset B $, então $ x \not \in (A - B) $.
  \end{enumerate}

\solution
  \begin{enumerate}[label=\arabic*.]
    \item Considerando a proposição: $x \in A \land (A - B) \subset (C \cap D) \land x \not \in D$. 

    Supomos que $x \in (A - B)$, ou seja, $x \in (C \cap D)$, pois $(A - B) \subset (C \cap D)$ e, portanto, $x \in D$. Essa suposição leva a um absurdo, pois sabemos que $x \not \in D$. 

    Ou seja, $x \not \in (A - B)$, então $x \in (A \cap B) = x \in B$.

    \item Considerando a proposição: $ x \in C \land (A \cap C) \subset B $.

    Se supormos $x \in (A - B)$, então $x \in A$ e $x \not \in B$. Entretanto, sabemos, pela proposição inicial, que $x \in C$. Se $x \in A$ e $x \in C$, então $x \in (A \cap C)$ e, por sua vez, $(A \cap C) \subset B$: então $x \in B$. Entretanto, isso contradiz a nossa suposição inicial!


    % A partir da proposição, sabemos que intervalo $(A \cap C) \subset B$ - ou seja, como $x \in C$, $x \in (A \cap C) = x \in B$. Entretanto, se supormos $x \in (A - B)$, então $x \not \in B$: contradição!

    Ou seja, $x \not \in (A - B)$.
  \end{enumerate}

\end{document}
