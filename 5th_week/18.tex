%=====================================================================
% jhwhw.cls
% Provide jhwhw.cls class
%=====================================================================

%=====================================================================
% Identification
%=====================================================================

\documentclass[12pt, letterpaper]{report}

\usepackage[utf8]{inputenc}

\usepackage{graphicx}
\usepackage{fancyhdr}
\usepackage[top=1in,bottom=1in,left=1in,right=1in]{geometry}
\usepackage{empheq}
\usepackage{ifthen}

\usepackage{enumitem}

%=====================================================================
% Commands
%=====================================================================

  \setlength{\headheight}{15pt}
  \lhead{Isadora S. G. Rodopoulos}\chead{MC358}\rhead{15 de Março, 2016}
  \lfoot{}\cfoot{\thepage}\rfoot{}
  \pagestyle{fancy}

\ifx\pdfoutput\undefined                         %LaTeX
  \RequirePackage[ps2pdf,bookmarks=true]{hyperref}
  \hypersetup{ %
    pdfauthor   = {\@author},
    pdftitle    = {\@title},
    pdfcreator  = {LaTeX with hyperref package},
    pdfproducer = {dvips + ps2pdf}
  }
\else                                            %PDFLaTeX
  \RequirePackage[pdftex,bookmarks=true]{hyperref}
  \hypersetup{ %
    pdfauthor   = {\@author},
    pdftitle    = {\@title},
    pdfcreator  = {LaTeX with hyperref package},
    pdfproducer = {dvips + ps2pdf}
  }
\pdfadjustspacing=1
\fi

% Set up counters for problems and subsections

\newcounter{ProblemNum}
\newcounter{SubProblemNum}[ProblemNum]

\renewcommand{\theProblemNum}{\arabic{ProblemNum}}
\renewcommand{\theSubProblemNum}{\alph{SubProblemNum}}


\newcommand*{\anyproblem}[1]{\newpage\subsection*{#1}}
\newcommand*{\problem}[1]{\stepcounter{ProblemNum} %
   \anyproblem{Questão #1}}
\newcommand*{\soln}[1]{\subsubsection*{#1}}
\newcommand*{\solution}{\soln{Solução}}
\renewcommand*{\part}{\stepcounter{SubProblemNum} %
  \soln{Parte (\theSubProblemNum)}}

\renewcommand{\theenumi}{(\alph{enumi})}
\renewcommand{\labelenumi}{\theenumi}
\renewcommand{\theenumii}{\roman{enumii}}

% \def\problemmark{}

% % Typesetting problems

% % \newcommand*{\prob}[1]{\newpage \noindent \textbf{\Large #1}}
% % \newcommand*{\problem}[1]{\stepcounter{ProblemNum} \prob{Problem %
% % \theProblemNum.}}
% % \newcommand*{\soln}[1]{\\ \noindent \textbf{\Large #1}}
% % \newcommand*{\solution}{\soln{Solution}}
% % \renewcommand*{\part}{\\ \noindent \stepcounter{SubProblemNum} %
% % \textbf{\Large Part (\theSubProblemNum)}}

% \newcommand\problem{\@startsection{problem}{1}{\z@}%
%                      {-3.25ex \@plus -1ex \@minus -.2ex}%
%                      {1.5ex \@plus .2ex}%
%                      {\normalfont\large\bfseries}}

\begin{document}

\problem{18.}
  Expresse o $ k $-ésimo termo da sequência 

  \begin{equation}
    S = 1 \cdot 3 \cdot 4 + 2 \cdot 5 \cdot 7 + 3 \cdot 7 \cdot 10 + 4 \cdot 9 \cdot 13 + \dots 
  \end{equation}

  como função de $ k $. Mostre então que a soma dos $ n $ primeiros elementos de $ S $ é dada pela fórmula

  \begin{equation}
    \cfrac{n}{6} (n + 1) (9n^2 + 19n + 8).
  \end{equation}

\solution
    O $k$-ésimo termo corresponde a $f(k) = k \cdot (2k + 1) \cdot (3k + 1)$. A soma dos n primeiros termos corresponde a $ \sum_{k = 1}^n k \cdot (2k + 1) \cdot (3k + 1) = \sum_{k = 1}^n (2k^2 + k) \cdot (3k + 1) = \sum_{k = 1}^n (6k^3 + 5k^2 + k)$. \\ \\

    Sabemos que $\sum_{k = 1}^n k = \cfrac{n(n+1)}{2}$, $\sum_{k = 1}^n k^2 = \cfrac{n(n+1)(2n+1)}{6}$ e $\sum_{k = 1}^n k^3 = \cfrac{n^2(n + 1)^2}{4}$. \\ \\

    Logo, 
    \begin{center}
    $\sum_{k = 1}^n (6k^3 + 5k^2 + k) = \sum_{k = 1}^n 6k^3 + \sum_{k = 1}^n 5k^2 + \sum_{k = 1}^n k =$ \\
    $ 3(\cfrac{n^2(n + 1)^2}{2}) + 5(\cfrac{n(n+1)(2n+1)}{6}) + \cfrac{n(n+1)}{2} =$ \\
    $ \cfrac{9n^2(n + 1)^2 + 5n(n+1)(2n+1)+3n(n+1)}{6} =$ \\
    $ \cfrac{n}{6}(n+1)(9n(n+1) + 5(2n+1)+ 3) =$ \\
    $ \cfrac{n}{6}(n+1)(9n^2 + 9n + 10n + 5 + 3) =$ \\
    $ \cfrac{n}{6}(n+1)(9n^2 + 19n + 8)$.
    \end{center}

\end{document}
