%=====================================================================
% jhwhw.cls
% Provide jhwhw.cls class
%=====================================================================

%=====================================================================
% Identification
%=====================================================================

\documentclass[12pt, letterpaper]{report}

\usepackage[utf8]{inputenc}

\usepackage{amssymb}

\usepackage{graphicx}
\usepackage{fancyhdr}
\usepackage[top=1in,bottom=1in,left=1in,right=1in]{geometry}
\usepackage{empheq}
\usepackage{ifthen}

\usepackage{enumitem}

%=====================================================================
% Commands
%=====================================================================

  \setlength{\headheight}{15pt}
  \lhead{Isadora S. G. Rodopoulos}\chead{MC358}\rhead{15 de Março, 2016}
  \lfoot{}\cfoot{\thepage}\rfoot{}
  \pagestyle{fancy}

\ifx\pdfoutput\undefined                         %LaTeX
  \RequirePackage[ps2pdf,bookmarks=true]{hyperref}
  \hypersetup{ %
    pdfauthor   = {\@author},
    pdftitle    = {\@title},
    pdfcreator  = {LaTeX with hyperref package},
    pdfproducer = {dvips + ps2pdf}
  }
\else                                            %PDFLaTeX
  \RequirePackage[pdftex,bookmarks=true]{hyperref}
  \hypersetup{ %
    pdfauthor   = {\@author},
    pdftitle    = {\@title},
    pdfcreator  = {LaTeX with hyperref package},
    pdfproducer = {dvips + ps2pdf}
  }
\pdfadjustspacing=1
\fi

% Set up counters for problems and subsections

\newcounter{ProblemNum}
\newcounter{SubProblemNum}[ProblemNum]

\renewcommand{\theProblemNum}{\arabic{ProblemNum}}
\renewcommand{\theSubProblemNum}{\alph{SubProblemNum}}


\newcommand*{\anyproblem}[1]{\newpage\subsection*{#1}}
\newcommand*{\problem}[1]{\stepcounter{ProblemNum} %
   \anyproblem{Questão #1}}
\newcommand*{\soln}[1]{\subsubsection*{#1}}
\newcommand*{\solution}{\soln{Solução}}
\renewcommand*{\part}{\stepcounter{SubProblemNum} %
  \soln{Parte (\theSubProblemNum)}}

\renewcommand{\theenumi}{(\alph{enumi})}
\renewcommand{\labelenumi}{\theenumi}
\renewcommand{\theenumii}{\roman{enumii}}

% \def\problemmark{}

% % Typesetting problems

% % \newcommand*{\prob}[1]{\newpage \noindent \textbf{\Large #1}}
% % \newcommand*{\problem}[1]{\stepcounter{ProblemNum} \prob{Problem %
% % \theProblemNum.}}
% % \newcommand*{\soln}[1]{\\ \noindent \textbf{\Large #1}}
% % \newcommand*{\solution}{\soln{Solution}}
% % \renewcommand*{\part}{\\ \noindent \stepcounter{SubProblemNum} %
% % \textbf{\Large Part (\theSubProblemNum)}}

% \newcommand\problem{\@startsection{problem}{1}{\z@}%
%                      {-3.25ex \@plus -1ex \@minus -.2ex}%
%                      {1.5ex \@plus .2ex}%
%                      {\normalfont\large\bfseries}}

\begin{document}

\problem{7.}
  \textbf{Prove ou disprove:} todo número ímpar pode ser escrito como a soma de um primo e o dobro de um quadrado. Dica: Escreva (e execute) um programa (simples) que verifica que 5777 é um contra-exemplo. Se você é um bom programador, seu programa só precisará testar a primalidade de 54 números inteiros.

  Antes de responder, considere qual a maneira mais sucinta de você apresentar sua resposta! \textit{Sugestão:} anexe um PDF com não mais de 54 linhas.

\solution
  A proposição é falsa, uma vez que o número $5777$ é um contra-exemplo: $\nexists a, b$ t.q. $a$ é primo e $a + 2b^2 = 5777$. A verificação foi feita através de um programa escrito em C, como pode ser conferida a seguir:

  Testing for a = 2 and b = 1 ... 53 FAILED! \\
  Testing for a = 3 and b = 1 ... 53 FAILED! \\
  Testing for a = 5 and b = 1 ... 53 FAILED! \\
  Testing for a = 7 and b = 1 ... 53 FAILED! \\
  Testing for a = 11 and b = 1 ... 53 FAILED! \\
  Testing for a = 13 and b = 1 ... 53 FAILED! \\
  Testing for a = 17 and b = 1 ... 53 FAILED! \\
  Testing for a = 19 and b = 1 ... 53 FAILED! \\
  Testing for a = 23 and b = 1 ... 53 FAILED! \\
  Testing for a = 29 and b = 1 ... 53 FAILED! \\
  Testing for a = 31 and b = 1 ... 53 FAILED! \\
  Testing for a = 37 and b = 1 ... 53 FAILED! \\
  Testing for a = 41 and b = 1 ... 53 FAILED! \\
  Testing for a = 43 and b = 1 ... 53 FAILED! \\
  Testing for a = 47 and b = 1 ... 53 FAILED! \\
  Testing for a = 53 and b = 1 ... 53 FAILED! \\
  Testing for a = 59 and b = 1 ... 53 FAILED! \\
  Testing for a = 61 and b = 1 ... 53 FAILED! \\
  Testing for a = 67 and b = 1 ... 53 FAILED! \\
  Testing for a = 71 and b = 1 ... 53 FAILED! \\
  Testing for a = 73 and b = 1 ... 53 FAILED! \\
  Testing for a = 79 and b = 1 ... 53 FAILED! \\
  Testing for a = 83 and b = 1 ... 53 FAILED! \\
  Testing for a = 89 and b = 1 ... 53 FAILED! \\
  Testing for a = 97 and b = 1 ... 53 FAILED! \\
  Testing for a = 101 and b = 1 ... 53 FAILED! \\
  Testing for a = 103 and b = 1 ... 53 FAILED! \\
  Testing for a = 107 and b = 1 ... 53 FAILED! \\
  Testing for a = 109 and b = 1 ... 53 FAILED! \\
  Testing for a = 113 and b = 1 ... 53 FAILED! \\
  Testing for a = 127 and b = 1 ... 53 FAILED! \\
  Testing for a = 131 and b = 1 ... 53 FAILED! \\
  Testing for a = 137 and b = 1 ... 53 FAILED! \\
  Testing for a = 139 and b = 1 ... 53 FAILED! \\
  Testing for a = 149 and b = 1 ... 53 FAILED! \\
  Testing for a = 151 and b = 1 ... 53 FAILED! \\
  Testing for a = 157 and b = 1 ... 53 FAILED! \\
  Testing for a = 163 and b = 1 ... 52 FAILED! \\
  Testing for a = 167 and b = 1 ... 52 FAILED! \\
  Testing for a = 173 and b = 1 ... 52 FAILED! \\
  Testing for a = 179 and b = 1 ... 52 FAILED! \\
  Testing for a = 181 and b = 1 ... 52 FAILED! \\
  Testing for a = 191 and b = 1 ... 52 FAILED! \\
  Testing for a = 193 and b = 1 ... 52 FAILED! \\
  Testing for a = 197 and b = 1 ... 52 FAILED! \\
  Testing for a = 199 and b = 1 ... 52 FAILED! \\
  Testing for a = 211 and b = 1 ... 52 FAILED! \\
  Testing for a = 223 and b = 1 ... 52 FAILED! \\
  Testing for a = 227 and b = 1 ... 52 FAILED! \\
  Testing for a = 229 and b = 1 ... 52 FAILED! \\
  Testing for a = 233 and b = 1 ... 52 FAILED! \\
  Testing for a = 239 and b = 1 ... 52 FAILED! \\
  Testing for a = 241 and b = 1 ... 52 FAILED! \\
  Testing for a = 251 and b = 1 ... 52 FAILED! \\

\end{document}