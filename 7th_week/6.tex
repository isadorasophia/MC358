%=====================================================================
% jhwhw.cls
% Provide jhwhw.cls class
%=====================================================================

%=====================================================================
% Identification
%=====================================================================

\documentclass[12pt, letterpaper]{report}

\usepackage[utf8]{inputenc}

\usepackage{graphicx}
\usepackage{fancyhdr}
\usepackage[top=1in,bottom=1in,left=1in,right=1in]{geometry}
\usepackage{empheq}
\usepackage{ifthen}

\usepackage{enumitem}

\usepackage{amssymb}
\newcommand*{\CQD}{\hfill\ensuremath{\blacksquare}}%


%=====================================================================
% Commands
%=====================================================================

  \setlength{\headheight}{15pt}
  \lhead{Isadora S. G. Rodopoulos}\chead{MC358}\rhead{19 de Abril, 2016}
  \lfoot{}\cfoot{\thepage}\rfoot{}
  \pagestyle{fancy}

\ifx\pdfoutput\undefined                         %LaTeX
  \RequirePackage[ps2pdf,bookmarks=true]{hyperref}
  \hypersetup{ %
    pdfauthor   = {\@author},
    pdftitle    = {\@title},
    pdfcreator  = {LaTeX with hyperref package},
    pdfproducer = {dvips + ps2pdf}
  }
\else                                            %PDFLaTeX
  \RequirePackage[pdftex,bookmarks=true]{hyperref}
  \hypersetup{ %
    pdfauthor   = {\@author},
    pdftitle    = {\@title},
    pdfcreator  = {LaTeX with hyperref package},
    pdfproducer = {dvips + ps2pdf}
  }
\pdfadjustspacing=1
\fi

% Set up counters for problems and subsections

\newcounter{ProblemNum}
\newcounter{SubProblemNum}[ProblemNum]

\renewcommand{\theProblemNum}{\arabic{ProblemNum}}
\renewcommand{\theSubProblemNum}{\alph{SubProblemNum}}


\newcommand*{\anyproblem}[1]{\newpage\subsection*{#1}}
\newcommand*{\problem}[1]{\stepcounter{ProblemNum} %
   \anyproblem{Questão #1}}
\newcommand*{\soln}[1]{\subsubsection*{#1}}
\newcommand*{\solution}{\soln{Solução}}
\renewcommand*{\part}{\stepcounter{SubProblemNum} %
  \soln{Parte (\theSubProblemNum)}}

\renewcommand{\theenumi}{(\alph{enumi})}
\renewcommand{\labelenumi}{\theenumi}
\renewcommand{\theenumii}{\roman{enumii}}

% \def\problemmark{}

% % Typesetting problems

% % \newcommand*{\prob}[1]{\newpage \noindent \textbf{\Large #1}}
% % \newcommand*{\problem}[1]{\stepcounter{ProblemNum} \prob{Problem %
% % \theProblemNum.}}
% % \newcommand*{\soln}[1]{\\ \noindent \textbf{\Large #1}}
% % \newcommand*{\solution}{\soln{Solution}}
% % \renewcommand*{\part}{\\ \noindent \stepcounter{SubProblemNum} %
% % \textbf{\Large Part (\theSubProblemNum)}}

% \newcommand\problem{\@startsection{problem}{1}{\z@}%
%                      {-3.25ex \@plus -1ex \@minus -.2ex}%
%                      {1.5ex \@plus .2ex}%
%                      {\normalfont\large\bfseries}}

\begin{document}

\problem{6.}
  \begin{enumerate}[label=\arabic*.]
    \item Prove por indução em $ n $ que a derivada de $ x^n $ em relação a $ x $ é igual a $ nx^{n-1} $ para qualquer inteiro $ n > 0$. Dica: utilize a regra do produto para derivadas.
    \item Prove por indução em $ n $ que a derivada de $ x^n $ em relação a $ x $ é igual a $ nx^{n-1} $ para qualquer inteiro $ n \leq 0 $, e $ x \neq 0 $.
  \end{enumerate}

\solution
  \begin{enumerate}[label=\arabic*.]
    \item Prova por indução em $n$:

    \textbf{Base:} Em $n = 1$, a derivada de $x^1$ em relação a $x$ é $1 \cdot x^{1 - 1} = 1$. \CQD

    \textbf{Hipótese:} A derivada de $ x^n $ em relação a $ x $ é igual a $ nx^{n-1} $ para qualquer inteiro $ n > 0$.

    \textbf{Passo:} Considere $x^{n + 1} = x \cdot x^n$. Utilizando a regra do produto para a derivada em $x$, temos que $\frac{d}{dx} (x \cdot x^n) = x \cdot \frac{d}{dx}(x^n) + \frac{d}{dx}(x) \cdot x^n$.

    Pela nossa hipótese, $x \cdot \frac{d}{dx}(x^n) + \frac{d}{dx}(x) \cdot x^n = x (nx^{n - 1}) + x^n = nx^n + x^n = (n + 1)x^n$. Ou seja, $\frac{d}{dx} x^{n + 1} = (n + 1)x^{(n + 1) - 1}$. \CQD

    \item Prova por forte em $n$:

    \textbf{Base:} Em $n = 0$, a derivada de $x^0$ em relação a $x$ é $0 \cdot x^{0 - 1} = 0$; 

    Em $n = -1$, a derivada de $x^{-1}$ em relação a $x$ é $-1 \cdot x^{(-1 - 1)} = -x^{-2}$. \CQD

    \textbf{Hipótese:} A derivada de $ x^n $ em relação a $ x $ é igual a $ nx^{n-1} $ para qualquer inteiro $ n \leq 0$ e $x \neq 0$.

    \textbf{Passo:} Considere $x^{n - 1} = \cfrac{x^n}{x}$. Utilizando a regra do quociente para a derivada em $x$, temos que $\frac{d}{dx} \cfrac{x^n}{x} = \cfrac{x \cdot \frac{d}{dx}(x^n) - \frac{d}{dx}(x) \cdot x^n}{x^2}$.

    Pela nossa hipótese, $\cfrac{x \cdot \frac{d}{dx}(x^n) - \frac{d}{dx}(x) \cdot x^n}{x^2} = \cfrac{x (nx^{n - 1}) - x^n}{x^2} = \cfrac{nx^n - x^n}{x^2} =$
    
    $x^{n - 2}(n - 1)$. Ou seja, $\frac{d}{dx} x^{n - 1} = (n - 1)x^{(n - 1) - 1}$. \CQD
  \end{enumerate}
\end{document}
