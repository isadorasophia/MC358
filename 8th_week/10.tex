%=====================================================================
% jhwhw.cls
% Provide jhwhw.cls class
%=====================================================================

%=====================================================================
% Identification
%=====================================================================

\documentclass[12pt, letterpaper]{report}

\usepackage[utf8]{inputenc}

\usepackage{graphicx}
\usepackage{fancyhdr}
\usepackage[top=1in,bottom=1in,left=1in,right=1in]{geometry}
\usepackage{empheq}
\usepackage{ifthen}

\usepackage{enumitem}

\usepackage{amssymb}
\newcommand*{\CQD}{\hfill\ensuremath{\blacksquare}}%


%=====================================================================
% Commands
%=====================================================================

  \setlength{\headheight}{15pt}
  \lhead{Isadora S. G. Rodopoulos}\chead{MC358}\rhead{03 de Maio, 2016}
  \lfoot{}\cfoot{\thepage}\rfoot{}
  \pagestyle{fancy}

\ifx\pdfoutput\undefined                         %LaTeX
  \RequirePackage[ps2pdf,bookmarks=true]{hyperref}
  \hypersetup{ %
    pdfauthor   = {\@author},
    pdftitle    = {\@title},
    pdfcreator  = {LaTeX with hyperref package},
    pdfproducer = {dvips + ps2pdf}
  }
\else                                            %PDFLaTeX
  \RequirePackage[pdftex,bookmarks=true]{hyperref}
  \hypersetup{ %
    pdfauthor   = {\@author},
    pdftitle    = {\@title},
    pdfcreator  = {LaTeX with hyperref package},
    pdfproducer = {dvips + ps2pdf}
  }
\pdfadjustspacing=1
\fi

% Set up counters for problems and subsections

\newcounter{ProblemNum}
\newcounter{SubProblemNum}[ProblemNum]

\renewcommand{\theProblemNum}{\arabic{ProblemNum}}
\renewcommand{\theSubProblemNum}{\alph{SubProblemNum}}


\newcommand*{\anyproblem}[1]{\newpage\subsection*{#1}}
\newcommand*{\problem}[1]{\stepcounter{ProblemNum} %
   \anyproblem{Questão #1}}
\newcommand*{\soln}[1]{\subsubsection*{#1}}
\newcommand*{\solution}{\soln{Solução}}
\renewcommand*{\part}{\stepcounter{SubProblemNum} %
  \soln{Parte (\theSubProblemNum)}}

\renewcommand{\theenumi}{(\alph{enumi})}
\renewcommand{\labelenumi}{\theenumi}
\renewcommand{\theenumii}{\roman{enumii}}

% \def\problemmark{}

% % Typesetting problems

% % \newcommand*{\prob}[1]{\newpage \noindent \textbf{\Large #1}}
% % \newcommand*{\problem}[1]{\stepcounter{ProblemNum} \prob{Problem %
% % \theProblemNum.}}
% % \newcommand*{\soln}[1]{\\ \noindent \textbf{\Large #1}}
% % \newcommand*{\solution}{\soln{Solution}}
% % \renewcommand*{\part}{\\ \noindent \stepcounter{SubProblemNum} %
% % \textbf{\Large Part (\theSubProblemNum)}}

% \newcommand\problem{\@startsection{problem}{1}{\z@}%
%                      {-3.25ex \@plus -1ex \@minus -.2ex}%
%                      {1.5ex \@plus .2ex}%
%                      {\normalfont\large\bfseries}}

\begin{document}

\problem{10.}
Seja $ n $ um inteiro positivo. Suponha que $ n $ cordas são traçadas num círculo de modo que qualquer par de cordas se cruza, mas nenhum trio delas se cruza num mesmo ponto. Prove que o círculo fica dividido por essas $ n $ cordas em $ {n^2 + n + 2} \over {2} $ regiões.

\solution
    Prova por indução em $n$:

    \textbf{Base:} Em $n = 1$, o círculo fica dividio em $\frac{1^2 + 1 + 2}{2} = 2$ regiões, o que é verdade. \CQD

    \textbf{Hipótese:} Um círculo com $n \geq 1$ cordas - de modo que qualquer par de cordas se cruza, mas nenhum trio delas se cruza num mesmo ponto - fica dividido em $ {n^2 + n + 2} \over {2} $ regiões.

    \textbf{Passo:} Considere um círculo com $n + 1$ cordas respeitando a propriedade:

    $P(n + 1) =$ qualquer par de cordas se cruza, mas nenhum trio delas se cruza num mesmo ponto. 

    Em seguida, retiramos uma corda, o que resulta em um círculo com $P(n)$ cordas. P.H.I., o círculo fica divido em $ {n^2 + n + 2} \over {2} $ regiões. Se recolocarmos a corda que retiramos no círculo, ele intercepta todas as outras cordas e forma $n + 1$ regiões a mais no círculo. 

    Ou seja, ele divide o círculo em um número de regiões igual a $P(n + 1) = \frac{n^2 + n + 2}{2} + n + 1 = \frac{n^2 + n + 2 + 2n + 2}{2} = \frac{(n^2 + 2n + 1) + (n + 1) + 2}{2} = \frac{(n + 1)^2 + (n + 1) + 2}{2} $. \CQD

\end{document}
