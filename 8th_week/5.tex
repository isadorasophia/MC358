%=====================================================================
% jhwhw.cls
% Provide jhwhw.cls class
%=====================================================================

%=====================================================================
% Identification
%=====================================================================

\documentclass[12pt, letterpaper]{report}

\usepackage[utf8]{inputenc}

\usepackage{graphicx}
\usepackage{fancyhdr}
\usepackage[top=1in,bottom=1in,left=1in,right=1in]{geometry}
\usepackage{empheq}
\usepackage{ifthen}

\usepackage{enumitem}

\usepackage{amssymb}

\newcommand*{\CQD}{\hfill\ensuremath{\blacksquare}}%

%=====================================================================
% Commands
%=====================================================================

  \setlength{\headheight}{15pt}
  \lhead{Isadora S. G. Rodopoulos}\chead{MC358}\rhead{03 de Maio, 2016}
  \lfoot{}\cfoot{\thepage}\rfoot{}
  \pagestyle{fancy}

\ifx\pdfoutput\undefined                         %LaTeX
  \RequirePackage[ps2pdf,bookmarks=true]{hyperref}
  \hypersetup{ %
    pdfauthor   = {\@author},
    pdftitle    = {\@title},
    pdfcreator  = {LaTeX with hyperref package},
    pdfproducer = {dvips + ps2pdf}
  }
\else                                            %PDFLaTeX
  \RequirePackage[pdftex,bookmarks=true]{hyperref}
  \hypersetup{ %
    pdfauthor   = {\@author},
    pdftitle    = {\@title},
    pdfcreator  = {LaTeX with hyperref package},
    pdfproducer = {dvips + ps2pdf}
  }
\pdfadjustspacing=1
\fi

% Set up counters for problems and subsections

\newcounter{ProblemNum}
\newcounter{SubProblemNum}[ProblemNum]

\renewcommand{\theProblemNum}{\arabic{ProblemNum}}
\renewcommand{\theSubProblemNum}{\alph{SubProblemNum}}


\newcommand*{\anyproblem}[1]{\newpage\subsection*{#1}}
\newcommand*{\problem}[1]{\stepcounter{ProblemNum} %
   \anyproblem{Questão #1}}
\newcommand*{\soln}[1]{\subsubsection*{#1}}
\newcommand*{\solution}{\soln{Solução}}
\renewcommand*{\part}{\stepcounter{SubProblemNum} %
  \soln{Parte (\theSubProblemNum)}}

\renewcommand{\theenumi}{(\alph{enumi})}
\renewcommand{\labelenumi}{\theenumi}
\renewcommand{\theenumii}{\roman{enumii}}

% \def\problemmark{}

% % Typesetting problems

% % \newcommand*{\prob}[1]{\newpage \noindent \textbf{\Large #1}}
% % \newcommand*{\problem}[1]{\stepcounter{ProblemNum} \prob{Problem %
% % \theProblemNum.}}
% % \newcommand*{\soln}[1]{\\ \noindent \textbf{\Large #1}}
% % \newcommand*{\solution}{\soln{Solution}}
% % \renewcommand*{\part}{\\ \noindent \stepcounter{SubProblemNum} %
% % \textbf{\Large Part (\theSubProblemNum)}}

% \newcommand\problem{\@startsection{problem}{1}{\z@}%
%                      {-3.25ex \@plus -1ex \@minus -.2ex}%
%                      {1.5ex \@plus .2ex}%
%                      {\normalfont\large\bfseries}}

\begin{document}

\problem{5.}
Os números de Fermat, em homenagem ao matemático francês Pierre Fermat (1601 - 1665), são definidos como $ F_n = 2^{(2^n)} + 1 $ para todo $ n $ natural. Fermat mostrou que $ F_0 $,  $ F_1 $, $ F_2 $, $ F_3 $, e $ F_4 $ são primos e conjecturou que todos os números de Fermat eram primos. No entanto, mais de cem anos depois, Euler mostrou que $ F_5 $ não é primo.

Prove que $ \forall n \ge 1 $,  $ F_n = (F_0 \cdot F_1 \cdot F_2 \cdots F_{n - 1}) + 2 $.

\solution
    Prova por indução em $n$:

    \textbf{Base:} Se $n = 1$, $F_1 = 2^{2^1} + 1 = $ \textbf{5} e $F_1 = (2^{2^0} + 1) + 2 = (2^1 + 1) + 2 = $ \textbf{5}. \CQD

    \textbf{Hipótese:} $ \forall n \ge 1 $,  $ F_n = (F_0 \cdot F_1 \cdot F_2 \cdots F_{n - 1}) + 2 $.

    \textbf{Passo:} Sabendo que $ F_{n + 1} = 2^{2^{n + 1}} + 1 = 2^{(2^n) \cdot 2} + 1 - 2 + 2= (2^{(2^n)} - 1)(2^{(2^n)} + 1) + 2 = $ \\
    $(2^{(2^n)} - 1 + 2 - 2)(F_n) + 2 = (F_n - 2)(F_n) + 2$. 

    Assim, P.H.I. $(F_n - 2)(F_n) + 2 = ((F_0 \cdot F_1 \cdot F_2 \cdots F_{n - 1}) + 2 - 2)(F_n) + 2 =$ \\
    $(F_0 \cdot F_1 \cdot F_2 \cdots F_{n - 1} \cdot F_n) + 2$.

    Logo, $ F_{n + 1} = (F_0 \cdot F_1 \cdot F_2 \cdots F_{n - 1} \cdot F_n) + 2$. \CQD

\end{document}
