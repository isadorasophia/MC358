%=====================================================================
% jhwhw.cls
% Provide jhwhw.cls class
%=====================================================================

%=====================================================================
% Identification
%=====================================================================

\documentclass[12pt, letterpaper]{report}

\usepackage[utf8]{inputenc}

\usepackage{graphicx}
\usepackage{fancyhdr}
\usepackage[top=1in,bottom=1in,left=1in,right=1in]{geometry}
\usepackage{empheq}
\usepackage{ifthen}

\usepackage{enumitem}

\usepackage{amssymb}
\newcommand*{\CQD}{\hfill\ensuremath{\blacksquare}}%


%=====================================================================
% Commands
%=====================================================================

  \setlength{\headheight}{15pt}
  \lhead{Isadora S. G. Rodopoulos}\chead{MC358}\rhead{03 de Maio, 2016}
  \lfoot{}\cfoot{\thepage}\rfoot{}
  \pagestyle{fancy}

\ifx\pdfoutput\undefined                         %LaTeX
  \RequirePackage[ps2pdf,bookmarks=true]{hyperref}
  \hypersetup{ %
    pdfauthor   = {\@author},
    pdftitle    = {\@title},
    pdfcreator  = {LaTeX with hyperref package},
    pdfproducer = {dvips + ps2pdf}
  }
\else                                            %PDFLaTeX
  \RequirePackage[pdftex,bookmarks=true]{hyperref}
  \hypersetup{ %
    pdfauthor   = {\@author},
    pdftitle    = {\@title},
    pdfcreator  = {LaTeX with hyperref package},
    pdfproducer = {dvips + ps2pdf}
  }
\pdfadjustspacing=1
\fi

% Set up counters for problems and subsections

\newcounter{ProblemNum}
\newcounter{SubProblemNum}[ProblemNum]

\renewcommand{\theProblemNum}{\arabic{ProblemNum}}
\renewcommand{\theSubProblemNum}{\alph{SubProblemNum}}


\newcommand*{\anyproblem}[1]{\newpage\subsection*{#1}}
\newcommand*{\problem}[1]{\stepcounter{ProblemNum} %
   \anyproblem{Questão #1}}
\newcommand*{\soln}[1]{\subsubsection*{#1}}
\newcommand*{\solution}{\soln{Solução}}
\renewcommand*{\part}{\stepcounter{SubProblemNum} %
  \soln{Parte (\theSubProblemNum)}}

\renewcommand{\theenumi}{(\alph{enumi})}
\renewcommand{\labelenumi}{\theenumi}
\renewcommand{\theenumii}{\roman{enumii}}

% \def\problemmark{}

% % Typesetting problems

% % \newcommand*{\prob}[1]{\newpage \noindent \textbf{\Large #1}}
% % \newcommand*{\problem}[1]{\stepcounter{ProblemNum} \prob{Problem %
% % \theProblemNum.}}
% % \newcommand*{\soln}[1]{\\ \noindent \textbf{\Large #1}}
% % \newcommand*{\solution}{\soln{Solution}}
% % \renewcommand*{\part}{\\ \noindent \stepcounter{SubProblemNum} %
% % \textbf{\Large Part (\theSubProblemNum)}}

% \newcommand\problem{\@startsection{problem}{1}{\z@}%
%                      {-3.25ex \@plus -1ex \@minus -.2ex}%
%                      {1.5ex \@plus .2ex}%
%                      {\normalfont\large\bfseries}}

\begin{document}

\problem{7.}
  Prove que um `tabuleiro' tridimensional de tamanho $ 2^n \times 2^n \times 2^n $, com um cubo de dimensões $ 1 \times 1 \times 1 $ removido, pode ser completamente coberto por cubos de dimensões $ 2 \times 2 \times 2 $ com um cubo $ 1 \times 1 \times 1 $ removido (ou seja, estes cubos contêm apenas 7 dos 8 cubos $ 1 \times 1 \times 1 $ que estariam presentes num cubo completo $ 2 \times 2 \times 2 $).
\solution
    Prova por indução forte em $n$:

    \textbf{Base:} Se $n = 1$, estamos dispostos de um tabuleiro de $ 2 \times 2 \times 2 $. Se retirarmos um cubo de dimensões $1 \times 1 \times 1$, restam exatamente os 7 cubos de $ 1 \times 1 \times 1 $ e o cubo é coberto.  \CQD

    \textbf{Hipótese:} Um `tabuleiro' tridimensional de tamanho $ 2^n \times 2^n \times 2^n $, com um cubo de dimensões $ 1 \times 1 \times 1 $ removido, pode ser completamente coberto por cubos de dimensões $ 2 \times 2 \times 2 $ com um cubo $ 1 \times 1 \times 1 $ removido.

    \textbf{Passo:} Queremos provar que a hipótese é válida para $n + 1$. Divide-se o tabuleiro de tamanho $T_{n + 1}$ em 8 tabuleiros de $T_n$. Em seguida, removemos um cubo de dimensões $ 1 \times 1 \times 1 $ de um dos $T_n$.

    Colocamos a peça de $ 2 \times 2 \times 2 $ com um cubo $ 1 \times 1 \times 1 $ removido cobrindo cada casa $ 1 \times 1 \times 1 $ dos $T_n$ restantes. Dessa forma, já que cada $T_n$ está com uma casa $1 \times 1 \times 1$ faltando, sabemos que
    elas são cobertas pela peça de 7 cubos P.H.I.. \CQD
\end{document}
