%=====================================================================
% jhwhw.cls
% Provide jhwhw.cls class
%=====================================================================

%=====================================================================
% Identification
%=====================================================================

\documentclass[12pt, letterpaper]{report}

\usepackage[utf8]{inputenc}

\usepackage{graphicx}
\usepackage{fancyhdr}
\usepackage[top=1in,bottom=1in,left=1in,right=1in]{geometry}
\usepackage{empheq}
\usepackage{ifthen}

\usepackage{enumitem}

\usepackage{amssymb}
\newcommand*{\CQD}{\hfill\ensuremath{\blacksquare}}%


%=====================================================================
% Commands
%=====================================================================

  \setlength{\headheight}{15pt}
  \lhead{Isadora S. G. Rodopoulos}\chead{MC358}\rhead{03 de Maio, 2016}
  \lfoot{}\cfoot{\thepage}\rfoot{}
  \pagestyle{fancy}

\ifx\pdfoutput\undefined                         %LaTeX
  \RequirePackage[ps2pdf,bookmarks=true]{hyperref}
  \hypersetup{ %
    pdfauthor   = {\@author},
    pdftitle    = {\@title},
    pdfcreator  = {LaTeX with hyperref package},
    pdfproducer = {dvips + ps2pdf}
  }
\else                                            %PDFLaTeX
  \RequirePackage[pdftex,bookmarks=true]{hyperref}
  \hypersetup{ %
    pdfauthor   = {\@author},
    pdftitle    = {\@title},
    pdfcreator  = {LaTeX with hyperref package},
    pdfproducer = {dvips + ps2pdf}
  }
\pdfadjustspacing=1
\fi

% Set up counters for problems and subsections

\newcounter{ProblemNum}
\newcounter{SubProblemNum}[ProblemNum]

\renewcommand{\theProblemNum}{\arabic{ProblemNum}}
\renewcommand{\theSubProblemNum}{\alph{SubProblemNum}}


\newcommand*{\anyproblem}[1]{\newpage\subsection*{#1}}
\newcommand*{\problem}[1]{\stepcounter{ProblemNum} %
   \anyproblem{Questão #1}}
\newcommand*{\soln}[1]{\subsubsection*{#1}}
\newcommand*{\solution}{\soln{Solução}}
\renewcommand*{\part}{\stepcounter{SubProblemNum} %
  \soln{Parte (\theSubProblemNum)}}

\renewcommand{\theenumi}{(\alph{enumi})}
\renewcommand{\labelenumi}{\theenumi}
\renewcommand{\theenumii}{\roman{enumii}}

% \def\problemmark{}

% % Typesetting problems

% % \newcommand*{\prob}[1]{\newpage \noindent \textbf{\Large #1}}
% % \newcommand*{\problem}[1]{\stepcounter{ProblemNum} \prob{Problem %
% % \theProblemNum.}}
% % \newcommand*{\soln}[1]{\\ \noindent \textbf{\Large #1}}
% % \newcommand*{\solution}{\soln{Solution}}
% % \renewcommand*{\part}{\\ \noindent \stepcounter{SubProblemNum} %
% % \textbf{\Large Part (\theSubProblemNum)}}

% \newcommand\problem{\@startsection{problem}{1}{\z@}%
%                      {-3.25ex \@plus -1ex \@minus -.2ex}%
%                      {1.5ex \@plus .2ex}%
%                      {\normalfont\large\bfseries}}

\begin{document}

\problem{8.}
Uma barra de chocolate consiste de $ n $ quadrados formando um padrão retangular:

    \begin{center}
    \includegraphics{figures/chocolate} 
    \end{center}

    Você pode quebrar a barra, mas sempre ao longo das linhas que separam os quadrados.

    Conjecture qual o menor número de quebras que produzem $ n $ quadrados separados.
    Prove sua conjectura por indução.
    Se a `barra' fosse conexa mas já não formasse um padrão retangular, devido à remoção de alguns quadrados de chocolate, sua prova ainda funcionaria ou não?

\solution
    \part O menor número de quebras para produzir os $n$ quadrados separados é $n - 1$.

    \part Prova por indução forte em $n$:

    \textbf{Base:} Em $n = 1$, a barra já está com o quadrado separado e deve-se quebrar $1 - 1 = 0$ vezes. \CQD

    \textbf{Hipótese:} O menor número de quebras para separar todos os quadrados de uma barra chocolate de $n$ quadrados é $n - 1$ para $n \geq 1$.

    \textbf{Passo:} Separe uma barra de chocolate de tamanho $n + 1$ em dois pedaços de tamanhos $n_1$ e $n_2$, de forma que $n_1 + n_2 = n + 1$. 

    P.H.I., o menor número de quebras para separar todos os quadrados da barra de tamanho $n_1$ é $n_1 - 1$, enquanto a de $n_2$ é $n_2 - 2$. Dessa forma, o total necessário para quebrar a barra de tamanho $n + 1$ é $1 + (n_1 - 1) + (n_2 - 1) = (n_1 + n_2) + 1 - 2 = (n + 1) - 1 = n$. \CQD

    \part A prova continuaria válida, uma vez que o passo permanece coerente - basta dividir a barra em dois.
\end{document}
