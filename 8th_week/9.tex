%=====================================================================
% jhwhw.cls
% Provide jhwhw.cls class
%=====================================================================

%=====================================================================
% Identification
%=====================================================================

\documentclass[12pt, letterpaper]{report}

\usepackage[utf8]{inputenc}

\usepackage{graphicx}
\usepackage{fancyhdr}
\usepackage[top=1in,bottom=1in,left=1in,right=1in]{geometry}
\usepackage{empheq}
\usepackage{ifthen}

\usepackage{enumitem}

\usepackage{amssymb}
\usepackage{amsmath}
\newcommand*{\CQD}{\hfill\ensuremath{\blacksquare}}%


%=====================================================================
% Commands
%=====================================================================

  \setlength{\headheight}{15pt}
  \lhead{Isadora S. G. Rodopoulos}\chead{MC358}\rhead{03 de Maio, 2016}
  \lfoot{}\cfoot{\thepage}\rfoot{}
  \pagestyle{fancy}

\ifx\pdfoutput\undefined                         %LaTeX
  \RequirePackage[ps2pdf,bookmarks=true]{hyperref}
  \hypersetup{ %
    pdfauthor   = {\@author},
    pdftitle    = {\@title},
    pdfcreator  = {LaTeX with hyperref package},
    pdfproducer = {dvips + ps2pdf}
  }
\else                                            %PDFLaTeX
  \RequirePackage[pdftex,bookmarks=true]{hyperref}
  \hypersetup{ %
    pdfauthor   = {\@author},
    pdftitle    = {\@title},
    pdfcreator  = {LaTeX with hyperref package},
    pdfproducer = {dvips + ps2pdf}
  }
\pdfadjustspacing=1
\fi

% Set up counters for problems and subsections

\newcounter{ProblemNum}
\newcounter{SubProblemNum}[ProblemNum]

\renewcommand{\theProblemNum}{\arabic{ProblemNum}}
\renewcommand{\theSubProblemNum}{\alph{SubProblemNum}}


\newcommand*{\anyproblem}[1]{\newpage\subsection*{#1}}
\newcommand*{\problem}[1]{\stepcounter{ProblemNum} %
   \anyproblem{Questão #1}}
\newcommand*{\soln}[1]{\subsubsection*{#1}}
\newcommand*{\solution}{\soln{Solução}}
\renewcommand*{\part}{\stepcounter{SubProblemNum} %
  \soln{Parte (\theSubProblemNum)}}

\renewcommand{\theenumi}{(\alph{enumi})}
\renewcommand{\labelenumi}{\theenumi}
\renewcommand{\theenumii}{\roman{enumii}}

% \def\problemmark{}

% % Typesetting problems

% % \newcommand*{\prob}[1]{\newpage \noindent \textbf{\Large #1}}
% % \newcommand*{\problem}[1]{\stepcounter{ProblemNum} \prob{Problem %
% % \theProblemNum.}}
% % \newcommand*{\soln}[1]{\\ \noindent \textbf{\Large #1}}
% % \newcommand*{\solution}{\soln{Solution}}
% % \renewcommand*{\part}{\\ \noindent \stepcounter{SubProblemNum} %
% % \textbf{\Large Part (\theSubProblemNum)}}

% \newcommand\problem{\@startsection{problem}{1}{\z@}%
%                      {-3.25ex \@plus -1ex \@minus -.2ex}%
%                      {1.5ex \@plus .2ex}%
%                      {\normalfont\large\bfseries}}

\begin{document}

\problem{9.}
Considere o seguinte jogo para duas pessoas. Coloca-se um número arbitrário de moedas sobre a mesa, e cada jogador, alternadamente, retira no mínimo 1 e no máximo 4 moedas da pilha. Quem retirar a última moeda perde o jogo.

Defina $ f(n) $ como 1 se existe uma estratégia ganhadora para o jogador da vez quando há $ n $ moedas na mesa, e 0 se existe uma estratégia ganhadora para seu adversário. Isto é, $ f(n) = 1 $ se, diante de $ n $ moedas, o jogador da vez consegue sempre vencer o jogo com alguma sequência de jogadas, independentemente das jogadas do adversário e $ f(n) = 0 $ se, diante de $ n $ moedas, o jogador da vez é incapaz de vencer por melhor que ele escolha suas jogadas.

Por exemplo, $ f(1) = 0 $, por definição; mas $ f(5)=1 $, pois, diante de 5 moedas, o jogador da vez consegue ganhar tirando 4 moedas.

Determine uma fórmula para $ f(n) $ e prove-a por indução.

\solution
  \part
  Considere as seguintes fórmulas:

\begin{equation}
   g(n) = (n - 1) \mod 5
\end{equation}

\begin{equation}
    f(g(n)) = 
    \begin{dcases}
      0,              & \text{if } g(n) = 0\\
      1,              & \text{otherwise}
    \end{dcases}
\end{equation}

  \part
  Prova por indução em $n$:

    \textbf{Base:} Se $n = 1$, temos que $g(1) = (1 - 1) \mod 5 = 0$, ou seja, $f(g(n)) = 0$ e o jogador perde.

    \textbf{Hipótese:} Dada a quantidade $n \geq 2$ de moedas, há uma estratégia ganhadora se $f(g(n)) = 1$, ou seja, se $(n) - 1$ não é divisível por 5.

    \textbf{Passo:} Se $5$ divide $(n + 1) - 1$ e $g(n + 1) = 0$: o jogador pode retirar $x$ moedas com $x \subset \{1, 2, 3, 4\}$ e o oponente será disposto com $n - x$ moedas; P.H.I., $f(g(n - x)) = 1$, pois $n - x$ não divisível por 5 e o oponente ganha o jogo.

    Se $5$ não divide $(n + 1) - 1$ e $g(n + 1) \not = 0$: o jogador deve retirar $x = g(n)$ moedas, de forma que o oponente será disposto com $n - x$ moedas e, P.H.I., $f(g(n - x)) = 0$, pois $n - x$ divisível por 5 e o oponente perde o jogo.

\end{document}
