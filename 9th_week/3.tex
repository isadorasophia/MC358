%=====================================================================
% jhwhw.cls
% Provide jhwhw.cls class
%=====================================================================

%=====================================================================
% Identification
%=====================================================================

\documentclass[12pt, letterpaper]{report}

\usepackage[utf8]{inputenc}

\usepackage{graphicx}
\usepackage{fancyhdr}
\usepackage[top=1in,bottom=1in,left=1in,right=1in]{geometry}
\usepackage{empheq}
\usepackage{ifthen}

\usepackage{enumitem}

\usepackage{mathrsfs}

\usepackage{amssymb}
\newcommand*{\CQD}{\hfill\ensuremath{\blacksquare}}%

%=====================================================================
% Commands
%=====================================================================

  \setlength{\headheight}{15pt}
  \lhead{Isadora S. G. Rodopoulos}\chead{MC358}\rhead{03 de Maio, 2016}
  \lfoot{}\cfoot{\thepage}\rfoot{}
  \pagestyle{fancy}

\ifx\pdfoutput\undefined                         %LaTeX
  \RequirePackage[ps2pdf,bookmarks=true]{hyperref}
  \hypersetup{ %
    pdfauthor   = {\@author},
    pdftitle    = {\@title},
    pdfcreator  = {LaTeX with hyperref package},
    pdfproducer = {dvips + ps2pdf}
  }
\else                                            %PDFLaTeX
  \RequirePackage[pdftex,bookmarks=true]{hyperref}
  \hypersetup{ %
    pdfauthor   = {\@author},
    pdftitle    = {\@title},
    pdfcreator  = {LaTeX with hyperref package},
    pdfproducer = {dvips + ps2pdf}
  }
\pdfadjustspacing=1
\fi

% Set up counters for problems and subsections

\newcounter{ProblemNum}
\newcounter{SubProblemNum}[ProblemNum]

\renewcommand{\theProblemNum}{\arabic{ProblemNum}}
\renewcommand{\theSubProblemNum}{\alph{SubProblemNum}}


\newcommand*{\anyproblem}[1]{\newpage\subsection*{#1}}
\newcommand*{\problem}[1]{\stepcounter{ProblemNum} %
   \anyproblem{Questão #1}}
\newcommand*{\soln}[1]{\subsubsection*{#1}}
\newcommand*{\solution}{\soln{Solução}}
\renewcommand*{\part}{\stepcounter{SubProblemNum} %
  \soln{Parte (\theSubProblemNum)}}

\renewcommand{\theenumi}{(\alph{enumi})}
\renewcommand{\labelenumi}{\theenumi}
\renewcommand{\theenumii}{\roman{enumii}}

% \def\problemmark{}

% % Typesetting problems

% % \newcommand*{\prob}[1]{\newpage \noindent \textbf{\Large #1}}
% % \newcommand*{\problem}[1]{\stepcounter{ProblemNum} \prob{Problem %
% % \theProblemNum.}}
% % \newcommand*{\soln}[1]{\\ \noindent \textbf{\Large #1}}
% % \newcommand*{\solution}{\soln{Solution}}
% % \renewcommand*{\part}{\\ \noindent \stepcounter{SubProblemNum} %
% % \textbf{\Large Part (\theSubProblemNum)}}

% \newcommand\problem{\@startsection{problem}{1}{\z@}%
%                      {-3.25ex \@plus -1ex \@minus -.2ex}%
%                      {1.5ex \@plus .2ex}%
%                      {\normalfont\large\bfseries}}

\begin{document}

\problem{3.}
  Prove por indução que a seguinte desigualdade é válida para qualquer $ n \geq 1 $:

  $ 1 + \frac{1}{2^2} + \frac{1}{3^2} + \dots + \frac{1}{n^2} < 2 $.

\solution
    \part
    Para provar por indução em $n$, é necessário tormar uma hipótese de indução mais forte. 

    Assim, considere que $\frac{1}{n^2} < \frac{1}{n(n - 1)} = \frac{1}{n - 1} - \frac{1}{n}$.

    Ou seja, $\frac{1}{2^2} + \dots + \frac{1}{n^2} \leq (\frac{1}{1} - \frac{1}{2}) + (\frac{1}{2} - \frac{1}{3}) + (\frac{1}{3} - \frac{1}{4}) + \dots + (\frac{1}{n - 1} - \frac{1}{n}) = 1 - \frac{1}{n}$.

    Logo $1 + \frac{1}{2^2} + \dots + \frac{1}{n^2} \leq 1 - \frac{1}{n} + 1 < 2$.

    Ou seja, provaremos que para todo $n \geq 1$, $ 1 + \frac{1}{2^2} + \frac{1}{3^2} + \dots + \frac{1}{n^2} \leq 2 - \frac{1}{n}$, usando indução em $n$:

    \part
    Prova por indução em $n$:

    \textbf{Base:} Em $n = 1$, temos que $1 \leq 2 - \frac{1}{1} = 1$. \CQD

    \textbf{Hipótese:} Para todo $n > 1$, a desigualdade $1 + \frac{1}{2^2} + \frac{1}{3^2} + \dots + \frac{1}{n^2} \leq 2 - \frac{1}{n} $ é válida.

    \textbf{Passo:} P.H.I., sabemos que $1 + \frac{1}{2^2} + \frac{1}{3^2} + \dots + \frac{1}{n^2} < 2 - \frac{1}{n}$. Se somarmos o termo $\frac{1}{(n + 1)^2}$, temos que:

    $1 + \frac{1}{2^2} + \frac{1}{3^2} + \dots + \frac{1}{n^2} + \frac{1}{(n + 1)^2} \leq 2 - \frac{1}{n} + \frac{1}{(n + 1)^2} < 2 - \frac{1}{n} + \frac{1}{n} \cdot \frac{1}{n + 1} =
       2 + \frac{1}{n} (- 1 + \frac{1}{n + 1}) = 2 + \frac{1}{n} (\frac{-(n + 1) + 1}{n+1}) = 2 + \frac{1}{n} \cdot \frac{-n}{n+1} = 2 - \frac{1}{n+1}.$

       Ou seja, $1 + \frac{1}{2^2} + \frac{1}{3^2} + \dots + \frac{1}{(n + 1)^2} < 2 - \frac{1}{n+1} < 2$. \CQD

\end{document}
